% Technical Report Template
% For corporate, institutional, or research documentation
%
% Compile with: pdflatex main.tex
%               bibtex main
%               pdflatex main.tex
%               pdflatex main.tex

\documentclass[11pt,a4paper]{report}

% Essential packages
\usepackage[margin=1in]{geometry}
\usepackage{times}
\usepackage{graphicx}
\usepackage{booktabs}
\usepackage{longtable}
\usepackage{amsmath}
\usepackage{amssymb}
\usepackage{siunitx}
\usepackage{float}
\usepackage{caption}
\usepackage{subcaption}
\usepackage{hyperref}
\usepackage{natbib}
\usepackage{setspace}
\usepackage{fancyhdr}
\usepackage{lastpage}
\usepackage{titlesec}
\usepackage{tocloft}
\usepackage{xcolor}
\usepackage{listings}
\usepackage{enumitem}
\usepackage[title]{appendix}

% Colors for code listings and links
\definecolor{codegreen}{rgb}{0,0.6,0}
\definecolor{codegray}{rgb}{0.5,0.5,0.5}
\definecolor{codepurple}{rgb}{0.58,0,0.82}
\definecolor{backcolour}{rgb}{0.95,0.95,0.92}
\definecolor{linkblue}{rgb}{0,0.4,0.7}

% Code listing style
\lstdefinestyle{mystyle}{
    backgroundcolor=\color{backcolour},
    commentstyle=\color{codegreen},
    keywordstyle=\color{codepurple},
    numberstyle=\tiny\color{codegray},
    stringstyle=\color{codegreen},
    basicstyle=\ttfamily\footnotesize,
    breakatwhitespace=false,
    breaklines=true,
    captionpos=b,
    keepspaces=true,
    numbers=left,
    numbersep=5pt,
    showspaces=false,
    showstringspaces=false,
    showtabs=false,
    tabsize=2
}
\lstset{style=mystyle}

% Hyperlink configuration
\hypersetup{
    colorlinks=true,
    linkcolor=linkblue,
    filecolor=linkblue,
    urlcolor=linkblue,
    citecolor=linkblue,
    pdftitle={{{TITLE}}},
    pdfauthor={{{AUTHOR}}}
}

% Header and footer configuration
\pagestyle{fancy}
\fancyhf{}
\fancyhead[L]{\leftmark}
\fancyhead[R]{{{REPORT_NUMBER}}}
\fancyfoot[C]{Page \thepage\ of \pageref{LastPage}}
\fancyfoot[R]{\footnotesize {{CLASSIFICATION}}}
\renewcommand{\headrulewidth}{0.4pt}
\renewcommand{\footrulewidth}{0.4pt}

% Chapter and section formatting
\titleformat{\chapter}[display]
    {\normalfont\huge\bfseries}{\chaptertitlename\ \thechapter}{20pt}{\Huge}
\titlespacing*{\chapter}{0pt}{0pt}{40pt}

% Line spacing
\onehalfspacing

% Document information
\newcommand{\reporttitle}{{{TITLE}}}
\newcommand{\reportsubtitle}{{{SUBTITLE}}}
\newcommand{\reportnumber}{{{REPORT_NUMBER}}}
\newcommand{\reportversion}{{{VERSION}}}
\newcommand{\reportdate}{{{DATE}}}
\newcommand{\reportauthor}{{{AUTHOR}}}

\begin{document}

% ===========================================
% TITLE PAGE
% ===========================================
\begin{titlepage}
    \centering

    % Organization logo
    \vspace*{1cm}
    % \includegraphics[width=0.3\textwidth]{figures/logo.png}
    \fbox{\parbox{0.3\textwidth}{\centering\vspace{1cm}[Organization Logo]\vspace{1cm}}}

    \vspace{1.5cm}

    % Organization name
    {\Large\textbf{{{ORGANIZATION}}}}\\[0.5cm]
    {\large {{DEPARTMENT}}}

    \vspace{2cm}

    % Report type
    {\large Technical Report}\\[0.3cm]
    {\large Report No.: \reportnumber}

    \vspace{1.5cm}

    % Title
    \rule{\textwidth}{1.5pt}\\[0.5cm]
    {\Huge\textbf{\reporttitle}}\\[0.3cm]
    {\Large \reportsubtitle}\\[0.5cm]
    \rule{\textwidth}{1.5pt}

    \vspace{2cm}

    % Author information
    \begin{tabular}{ll}
        \textbf{Prepared by:} & {{AUTHOR}} \\
        \textbf{Email:} & {{AUTHOR_EMAIL}} \\
        \textbf{Reviewed by:} & {{REVIEWER}} \\
        \textbf{Approved by:} & {{APPROVER}} \\
    \end{tabular}

    \vfill

    % Version and date
    \begin{tabular}{ll}
        \textbf{Version:} & \reportversion \\
        \textbf{Date:} & \reportdate \\
        \textbf{Classification:} & {{CLASSIFICATION}} \\
    \end{tabular}

    \vspace{1cm}

    % Footer
    {\small {{ORGANIZATION_ADDRESS}}}

\end{titlepage}

% ===========================================
% DOCUMENT CONTROL
% ===========================================
\chapter*{Document Control}
\addcontentsline{toc}{chapter}{Document Control}

\section*{Revision History}

\begin{longtable}{|c|c|p{6cm}|c|}
\hline
\textbf{Version} & \textbf{Date} & \textbf{Description} & \textbf{Author} \\
\hline
\endfirsthead
\hline
\textbf{Version} & \textbf{Date} & \textbf{Description} & \textbf{Author} \\
\hline
\endhead
\hline
\endfoot
\hline
\endlastfoot
1.0 & {{INITIAL_DATE}} & Initial release & {{AUTHOR}} \\
\hline
{{VERSION}} & {{DATE}} & {{VERSION_DESCRIPTION}} & {{AUTHOR}} \\
\hline
\end{longtable}

\section*{Distribution List}

\begin{tabular}{|l|l|l|}
\hline
\textbf{Name} & \textbf{Organization} & \textbf{Role} \\
\hline
{{RECIPIENT_1}} & {{RECIPIENT_1_ORG}} & {{RECIPIENT_1_ROLE}} \\
\hline
{{RECIPIENT_2}} & {{RECIPIENT_2_ORG}} & {{RECIPIENT_2_ROLE}} \\
\hline
{{RECIPIENT_3}} & {{RECIPIENT_3_ORG}} & {{RECIPIENT_3_ROLE}} \\
\hline
\end{tabular}

\section*{Referenced Documents}

\begin{enumerate}
    \item {{REFERENCE_DOC_1}}
    \item {{REFERENCE_DOC_2}}
    \item {{REFERENCE_DOC_3}}
\end{enumerate}

% ===========================================
% TABLE OF CONTENTS
% ===========================================
\tableofcontents
\listoffigures
\listoftables

% ===========================================
% EXECUTIVE SUMMARY
% ===========================================
\chapter*{Executive Summary}
\addcontentsline{toc}{chapter}{Executive Summary}

{{EXECUTIVE_SUMMARY}}

\textbf{Key Findings:}
\begin{itemize}
    \item {{KEY_FINDING_1}}
    \item {{KEY_FINDING_2}}
    \item {{KEY_FINDING_3}}
\end{itemize}

\textbf{Recommendations:}
\begin{enumerate}
    \item {{RECOMMENDATION_1}}
    \item {{RECOMMENDATION_2}}
    \item {{RECOMMENDATION_3}}
\end{enumerate}

% ===========================================
% CHAPTER 1: INTRODUCTION
% ===========================================
\chapter{Introduction}
\label{chap:introduction}

\section{Purpose}

The purpose of this technical report is to {{REPORT_PURPOSE}}. This document provides {{DOCUMENT_PROVIDES}}.

\section{Scope}

This report covers {{SCOPE_DESCRIPTION}}. The following topics are addressed:

\begin{itemize}
    \item {{SCOPE_ITEM_1}}
    \item {{SCOPE_ITEM_2}}
    \item {{SCOPE_ITEM_3}}
    \item {{SCOPE_ITEM_4}}
\end{itemize}

\section{Background}

{{BACKGROUND_DESCRIPTION}}

The project was initiated in response to {{PROJECT_MOTIVATION}}. Previous work in this area includes {{PREVIOUS_WORK}} \citep{smith2023technical}.

\section{Objectives}

The primary objectives of this work are:

\begin{enumerate}
    \item {{OBJECTIVE_1}}
    \item {{OBJECTIVE_2}}
    \item {{OBJECTIVE_3}}
\end{enumerate}

\section{Document Organization}

This report is organized as follows:
\begin{itemize}
    \item \textbf{Chapter~\ref{chap:introduction}}: Introduction and background
    \item \textbf{Chapter~\ref{chap:methodology}}: Technical approach and methodology
    \item \textbf{Chapter~\ref{chap:analysis}}: Analysis and findings
    \item \textbf{Chapter~\ref{chap:results}}: Results and discussion
    \item \textbf{Chapter~\ref{chap:conclusions}}: Conclusions and recommendations
\end{itemize}

% ===========================================
% CHAPTER 2: METHODOLOGY
% ===========================================
\chapter{Technical Approach}
\label{chap:methodology}

\section{Overview}

This chapter describes the methodology used to {{METHODOLOGY_PURPOSE}}. The approach consists of the following phases:

\begin{enumerate}
    \item {{PHASE_1}}
    \item {{PHASE_2}}
    \item {{PHASE_3}}
\end{enumerate}

\section{Requirements Analysis}

Table~\ref{tab:requirements} summarizes the key requirements addressed in this work.

\begin{table}[htbp]
    \centering
    \caption{Summary of technical requirements.}
    \label{tab:requirements}
    \begin{tabular}{clcc}
        \toprule
        \textbf{ID} & \textbf{Requirement} & \textbf{Priority} & \textbf{Status} \\
        \midrule
        REQ-001 & {{REQUIREMENT_1}} & High & Complete \\
        REQ-002 & {{REQUIREMENT_2}} & High & Complete \\
        REQ-003 & {{REQUIREMENT_3}} & Medium & In Progress \\
        REQ-004 & {{REQUIREMENT_4}} & Low & Planned \\
        \bottomrule
    \end{tabular}
\end{table}

\section{System Architecture}

Figure~\ref{fig:architecture} illustrates the overall system architecture.

\begin{figure}[htbp]
    \centering
    % \includegraphics[width=0.9\textwidth]{figures/architecture.png}
    \fbox{\parbox{0.9\textwidth}{\centering\vspace{4cm}[System Architecture Diagram]\vspace{4cm}}}
    \caption{High-level system architecture showing {{ARCHITECTURE_DESCRIPTION}}.}
    \label{fig:architecture}
\end{figure}

\section{Design Considerations}

\subsection{Performance Requirements}

The system must meet the following performance criteria:
\begin{itemize}
    \item {{PERFORMANCE_REQ_1}}
    \item {{PERFORMANCE_REQ_2}}
    \item {{PERFORMANCE_REQ_3}}
\end{itemize}

\subsection{Security Considerations}

Security measures implemented include:
\begin{itemize}
    \item {{SECURITY_MEASURE_1}}
    \item {{SECURITY_MEASURE_2}}
    \item {{SECURITY_MEASURE_3}}
\end{itemize}

\section{Tools and Technologies}

The following tools and technologies were used:

\begin{table}[htbp]
    \centering
    \caption{Tools and technologies used in this project.}
    \label{tab:tools}
    \begin{tabular}{lll}
        \toprule
        \textbf{Category} & \textbf{Tool} & \textbf{Version} \\
        \midrule
        {{TOOL_CATEGORY_1}} & {{TOOL_1}} & {{TOOL_1_VERSION}} \\
        {{TOOL_CATEGORY_2}} & {{TOOL_2}} & {{TOOL_2_VERSION}} \\
        {{TOOL_CATEGORY_3}} & {{TOOL_3}} & {{TOOL_3_VERSION}} \\
        \bottomrule
    \end{tabular}
\end{table}

% ===========================================
% CHAPTER 3: ANALYSIS
% ===========================================
\chapter{Analysis and Findings}
\label{chap:analysis}

\section{Data Collection}

Data was collected from {{DATA_SOURCES}}. The dataset includes:

\begin{itemize}
    \item {{DATA_DESCRIPTION_1}}
    \item {{DATA_DESCRIPTION_2}}
    \item {{DATA_DESCRIPTION_3}}
\end{itemize}

\section{Analysis Methods}

The analysis employed the following methods:

\subsection{Quantitative Analysis}

{{QUANTITATIVE_ANALYSIS_DESCRIPTION}}

The key metrics used for evaluation include:
\begin{equation}
    {{METRIC_NAME}} = {{METRIC_FORMULA}}
    \label{eq:metric}
\end{equation}

\subsection{Qualitative Analysis}

{{QUALITATIVE_ANALYSIS_DESCRIPTION}}

\section{Findings}

\subsection{Finding 1: {{FINDING_1_TITLE}}}

{{FINDING_1_DESCRIPTION}}

\subsection{Finding 2: {{FINDING_2_TITLE}}}

{{FINDING_2_DESCRIPTION}}

\subsection{Finding 3: {{FINDING_3_TITLE}}}

{{FINDING_3_DESCRIPTION}}

% ===========================================
% CHAPTER 4: RESULTS
% ===========================================
\chapter{Results and Discussion}
\label{chap:results}

\section{Summary of Results}

Table~\ref{tab:results} presents a summary of the key results.

\begin{table}[htbp]
    \centering
    \caption{Summary of key results.}
    \label{tab:results}
    \begin{tabular}{lcc}
        \toprule
        \textbf{Metric} & \textbf{Target} & \textbf{Achieved} \\
        \midrule
        {{METRIC_1}} & {{TARGET_1}} & {{ACHIEVED_1}} \\
        {{METRIC_2}} & {{TARGET_2}} & {{ACHIEVED_2}} \\
        {{METRIC_3}} & {{TARGET_3}} & {{ACHIEVED_3}} \\
        \bottomrule
    \end{tabular}
\end{table}

\section{Performance Evaluation}

Figure~\ref{fig:performance} shows the performance comparison.

\begin{figure}[htbp]
    \centering
    % \includegraphics[width=0.8\textwidth]{figures/performance.png}
    \fbox{\parbox{0.8\textwidth}{\centering\vspace{4cm}[Performance Comparison Chart]\vspace{4cm}}}
    \caption{Performance comparison showing {{PERFORMANCE_DESCRIPTION}}.}
    \label{fig:performance}
\end{figure}

\section{Discussion}

\subsection{Interpretation of Results}

{{RESULTS_INTERPRETATION}}

\subsection{Comparison with Previous Work}

Compared to previous approaches \citep{johnson2022analysis}, our method demonstrates:
\begin{itemize}
    \item {{COMPARISON_1}}
    \item {{COMPARISON_2}}
    \item {{COMPARISON_3}}
\end{itemize}

\subsection{Limitations}

The following limitations should be considered:
\begin{enumerate}
    \item {{LIMITATION_1}}
    \item {{LIMITATION_2}}
    \item {{LIMITATION_3}}
\end{enumerate}

% ===========================================
% CHAPTER 5: CONCLUSIONS
% ===========================================
\chapter{Conclusions and Recommendations}
\label{chap:conclusions}

\section{Conclusions}

This technical report has presented {{CONCLUSION_SUMMARY}}.

The key conclusions are:
\begin{enumerate}
    \item {{CONCLUSION_1}}
    \item {{CONCLUSION_2}}
    \item {{CONCLUSION_3}}
\end{enumerate}

\section{Recommendations}

Based on the findings of this study, we recommend:

\subsection{Immediate Actions}
\begin{enumerate}
    \item {{IMMEDIATE_ACTION_1}}
    \item {{IMMEDIATE_ACTION_2}}
\end{enumerate}

\subsection{Long-term Actions}
\begin{enumerate}
    \item {{LONGTERM_ACTION_1}}
    \item {{LONGTERM_ACTION_2}}
\end{enumerate}

\section{Future Work}

The following areas warrant further investigation:
\begin{itemize}
    \item {{FUTURE_WORK_1}}
    \item {{FUTURE_WORK_2}}
    \item {{FUTURE_WORK_3}}
\end{itemize}

% ===========================================
% REFERENCES
% ===========================================
\bibliographystyle{plain}
\bibliography{references}

% ===========================================
% APPENDICES
% ===========================================
\begin{appendices}

\chapter{Technical Specifications}
\label{app:specs}

\section{System Specifications}

\begin{table}[htbp]
    \centering
    \caption{Detailed system specifications.}
    \begin{tabular}{ll}
        \toprule
        \textbf{Parameter} & \textbf{Value} \\
        \midrule
        {{SPEC_PARAM_1}} & {{SPEC_VALUE_1}} \\
        {{SPEC_PARAM_2}} & {{SPEC_VALUE_2}} \\
        {{SPEC_PARAM_3}} & {{SPEC_VALUE_3}} \\
        \bottomrule
    \end{tabular}
\end{table}

\chapter{Code Listings}
\label{app:code}

\begin{lstlisting}[language=Python, caption={Example code implementation.}]
# {{CODE_DESCRIPTION}}
def example_function(parameter):
    """
    {{FUNCTION_DESCRIPTION}}
    """
    result = process(parameter)
    return result
\end{lstlisting}

\chapter{Glossary}
\label{app:glossary}

\begin{description}
    \item[{{TERM_1}}] {{TERM_1_DEFINITION}}
    \item[{{TERM_2}}] {{TERM_2_DEFINITION}}
    \item[{{TERM_3}}] {{TERM_3_DEFINITION}}
\end{description}

\chapter{Acronyms}
\label{app:acronyms}

\begin{tabular}{ll}
    \textbf{{{ACRONYM_1}}} & {{ACRONYM_1_FULL}} \\
    \textbf{{{ACRONYM_2}}} & {{ACRONYM_2_FULL}} \\
    \textbf{{{ACRONYM_3}}} & {{ACRONYM_3_FULL}} \\
\end{tabular}

\end{appendices}

\end{document}
