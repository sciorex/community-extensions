\chapter{Results and Discussion}
\label{ch:results}

This chapter presents the experimental evaluation of the proposed methodology and discusses the findings in the context of prior work.

\section{Experimental Setup}
\label{sec:res-setup}

A comprehensive experimental framework was established to evaluate the proposed approach.

\subsection{Datasets}
\label{subsec:res-datasets}

The evaluation was conducted on three benchmark datasets:

\begin{table}[h]
\centering
\caption{Dataset characteristics.}
\label{tab:res-datasets}
\begin{tabular}{@{}lcccc@{}}
\toprule
Dataset & Samples & Features & Classes & Domain \\
\midrule
Dataset A & 10,000 & 256 & 10 & General \\
Dataset B & 50,000 & 512 & 100 & Specialized \\
Dataset C & 5,000 & 128 & 5 & Real-world \\
\bottomrule
\end{tabular}
\end{table}

\subsection{Baseline Methods}
\label{subsec:res-baselines}

The proposed method was compared against several baseline approaches:

\begin{enumerate}
    \item \textbf{Baseline A} \cite{chen2022foundations}: A classical approach with strong theoretical guarantees
    \item \textbf{Baseline B} \cite{smith2020classical}: A modern data-driven method
    \item \textbf{Baseline C} \cite{williams2024sota}: The current state-of-the-art approach
\end{enumerate}

\subsection{Evaluation Metrics}
\label{subsec:res-metrics}

Performance was evaluated using standard metrics:

\begin{itemize}
    \item \textbf{Accuracy}: Overall classification accuracy
    \item \textbf{F1 Score}: Harmonic mean of precision and recall
    \item \textbf{AUC-ROC}: Area under the receiver operating characteristic curve
    \item \textbf{Runtime}: Computational time in seconds
\end{itemize}

\section{Main Results}
\label{sec:res-main}

This section presents the primary experimental results.

\subsection{Performance Comparison}
\label{subsec:res-performance}

Table~\ref{tab:res-main} summarizes the performance across all datasets and methods.

\begin{table}[h]
\centering
\caption{Performance comparison (mean $\pm$ std over 5 runs).}
\label{tab:res-main}
\begin{tabular}{@{}llcccc@{}}
\toprule
Dataset & Method & Accuracy & F1 Score & AUC & Time (s) \\
\midrule
\multirow{4}{*}{A}
    & Baseline A & 0.821 $\pm$ 0.012 & 0.808 & 0.891 & 12.3 \\
    & Baseline B & 0.856 $\pm$ 0.009 & 0.843 & 0.912 & 45.7 \\
    & Baseline C & 0.878 $\pm$ 0.007 & 0.869 & 0.934 & 89.2 \\
    & \textbf{Proposed} & \textbf{0.892 $\pm$ 0.006} & \textbf{0.885} & \textbf{0.948} & \textbf{34.1} \\
\midrule
\multirow{4}{*}{B}
    & Baseline A & 0.756 $\pm$ 0.015 & 0.741 & 0.845 & 58.4 \\
    & Baseline B & 0.812 $\pm$ 0.011 & 0.798 & 0.889 & 187.3 \\
    & Baseline C & 0.845 $\pm$ 0.008 & 0.831 & 0.912 & 342.1 \\
    & \textbf{Proposed} & \textbf{0.867 $\pm$ 0.007} & \textbf{0.856} & \textbf{0.931} & \textbf{156.8} \\
\midrule
\multirow{4}{*}{C}
    & Baseline A & 0.798 $\pm$ 0.018 & 0.782 & 0.867 & 5.2 \\
    & Baseline B & 0.834 $\pm$ 0.014 & 0.821 & 0.901 & 18.9 \\
    & Baseline C & 0.861 $\pm$ 0.010 & 0.852 & 0.923 & 37.4 \\
    & \textbf{Proposed} & \textbf{0.879 $\pm$ 0.009} & \textbf{0.871} & \textbf{0.941} & \textbf{15.6} \\
\bottomrule
\end{tabular}
\end{table}

The proposed method consistently outperforms all baselines across datasets and metrics.

\subsection{Statistical Analysis}
\label{subsec:res-statistical}

Statistical significance was assessed using paired t-tests with Bonferroni correction. All improvements over the state-of-the-art (Baseline C) are statistically significant at $p < 0.01$.

\section{Ablation Study}
\label{sec:res-ablation}

An ablation study was conducted to understand the contribution of each component.

\begin{table}[h]
\centering
\caption{Ablation study results on Dataset A.}
\label{tab:res-ablation}
\begin{tabular}{@{}lcc@{}}
\toprule
Configuration & Accuracy & $\Delta$ \\
\midrule
Full model & 0.892 & --- \\
w/o Component 1 & 0.867 & -0.025 \\
w/o Component 2 & 0.854 & -0.038 \\
w/o Component 3 & 0.871 & -0.021 \\
w/o Regularization & 0.878 & -0.014 \\
\bottomrule
\end{tabular}
\end{table}

Component 2 contributes most significantly to the overall performance.

\section{Sensitivity Analysis}
\label{sec:res-sensitivity}

The sensitivity of the method to key hyperparameters was analyzed.

\begin{figure}[h]
    \centering
    %% Replace with actual figure
    \fbox{\parbox{0.8\textwidth}{\centering\vspace{2cm}[Sensitivity Analysis Plot]\vspace{2cm}}}
    \caption{Performance sensitivity to learning rate.}
    \label{fig:res-sensitivity}
\end{figure}

The method shows robust performance across a range of hyperparameter values, with optimal performance achieved at $\eta = 0.001$.

\section{Discussion}
\label{sec:res-discussion}

\subsection{Analysis of Results}
\label{subsec:res-analysis}

The experimental results demonstrate several key findings:

\begin{enumerate}
    \item The proposed method achieves superior accuracy compared to all baselines
    \item Computational efficiency is improved compared to the state-of-the-art
    \item Performance gains are consistent across diverse datasets
    \item The method shows robust behavior across hyperparameter settings
\end{enumerate}

\subsection{Comparison with Prior Work}
\label{subsec:res-comparison}

Compared to prior work, the proposed approach offers:

\begin{itemize}
    \item 1.6--2.2\% improvement in accuracy over the state-of-the-art
    \item 2--2.5$\times$ speedup in runtime compared to Baseline C
    \item Better generalization to real-world datasets (Dataset C)
\end{itemize}

\subsection{Limitations}
\label{subsec:res-limitations}

While the results are promising, several limitations should be acknowledged:

\begin{itemize}
    \item The evaluation is limited to three datasets; broader validation is needed
    \item The method requires careful hyperparameter tuning for optimal performance
    \item Scalability to very large datasets has not been fully tested
\end{itemize}

\section{Summary}
\label{sec:res-summary}

This chapter has presented comprehensive experimental results demonstrating:

\begin{itemize}
    \item Superior performance of the proposed method across multiple metrics
    \item Statistical significance of improvements over baselines
    \item Robustness of the approach through ablation and sensitivity analyses
    \item Honest acknowledgment of current limitations
\end{itemize}

The next chapter concludes the thesis and outlines directions for future work.
