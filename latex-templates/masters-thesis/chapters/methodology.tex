\chapter{Methodology}
\label{ch:methodology}

This chapter presents the proposed methodology in detail. The approach is designed to address the limitations identified in Chapter~\ref{ch:background} while maintaining practical applicability.

\section{Overview}
\label{sec:meth-overview}

The proposed methodology consists of three main components:

\begin{enumerate}
    \item Data preprocessing and feature extraction
    \item Core algorithm design
    \item Output generation and validation
\end{enumerate}

Figure~\ref{fig:meth-overview} provides a schematic overview of the complete pipeline.

\begin{figure}[h]
    \centering
    %% Replace with actual figure
    \fbox{\parbox{0.8\textwidth}{\centering\vspace{2cm}[Methodology Overview Diagram]\vspace{2cm}}}
    \caption{Overview of the proposed methodology.}
    \label{fig:meth-overview}
\end{figure}

\section{Data Preprocessing}
\label{sec:meth-preprocessing}

The preprocessing stage prepares raw data for subsequent analysis. This involves several steps to ensure data quality and consistency.

\subsection{Data Collection}
\label{subsec:meth-collection}

Data was collected from multiple sources to ensure diversity and representativeness. The collection process followed established protocols to maintain data integrity.

\subsection{Cleaning and Normalization}
\label{subsec:meth-cleaning}

Raw data often contains noise, missing values, and inconsistencies that must be addressed before analysis. The cleaning procedures include:

\begin{itemize}
    \item Removal of duplicate entries
    \item Handling of missing values through imputation
    \item Outlier detection and treatment
    \item Feature normalization to standard scales
\end{itemize}

\subsection{Feature Extraction}
\label{subsec:meth-features}

Feature extraction transforms raw data into a representation suitable for the core algorithm. The extracted features capture the essential characteristics of the data while reducing dimensionality.

\section{Proposed Algorithm}
\label{sec:meth-algorithm}

The core of the proposed methodology is a novel algorithm that addresses the identified research gaps. This section describes the algorithm in detail.

\subsection{Problem Formulation}
\label{subsec:meth-formulation}

The problem can be formulated as an optimization task:

\begin{equation}
    \min_{\theta} \mathcal{L}(\theta) = \frac{1}{N} \sum_{i=1}^{N} \ell(f_\theta(x_i), y_i) + \lambda R(\theta)
    \label{eq:meth-objective}
\end{equation}

where:
\begin{itemize}
    \item $\theta$ represents the model parameters
    \item $\ell(\cdot, \cdot)$ is the loss function
    \item $f_\theta(\cdot)$ is the model function
    \item $R(\theta)$ is a regularization term
    \item $\lambda$ controls the regularization strength
\end{itemize}

\subsection{Algorithm Design}
\label{subsec:meth-design}

The proposed algorithm iteratively updates the parameters to minimize the objective function. Algorithm~\ref{alg:proposed} presents the pseudocode.

\begin{algorithm}[h]
\caption{Proposed Algorithm}
\label{alg:proposed}
\begin{algorithmic}[1]
\REQUIRE Data $\mathcal{D} = \{(x_i, y_i)\}_{i=1}^N$, learning rate $\eta$, iterations $T$
\ENSURE Optimized parameters $\theta^*$
\STATE Initialize parameters $\theta_0$
\FOR{$t = 1$ to $T$}
    \STATE Compute gradient: $g_t \leftarrow \nabla_\theta \mathcal{L}(\theta_{t-1})$
    \STATE Update parameters: $\theta_t \leftarrow \theta_{t-1} - \eta \cdot g_t$
    \IF{convergence criterion met}
        \STATE \textbf{break}
    \ENDIF
\ENDFOR
\RETURN $\theta_T$
\end{algorithmic}
\end{algorithm}

\subsection{Theoretical Analysis}
\label{subsec:meth-theory}

This section provides theoretical justification for the proposed algorithm.

\begin{lemma}[Gradient Bound]
Under assumption A1, the gradient is bounded: $\|g_t\| \leq G$ for all $t$.
\end{lemma}

\begin{theorem}[Convergence Guarantee]
Given the gradient bound and a suitable learning rate schedule, the algorithm converges to a stationary point at a rate of $O(1/\sqrt{T})$.
\end{theorem}

\section{Implementation Details}
\label{sec:meth-implementation}

This section provides practical details necessary for implementing the proposed methodology.

\subsection{Software and Hardware}
\label{subsec:meth-software}

The implementation was developed using:
\begin{itemize}
    \item Programming language: Python 3.9
    \item Deep learning framework: PyTorch 2.0
    \item Numerical computing: NumPy, SciPy
    \item Hardware: NVIDIA RTX 3090 GPU, 64GB RAM
\end{itemize}

\subsection{Hyperparameter Settings}
\label{subsec:meth-hyperparams}

The hyperparameters were selected through cross-validation:

\begin{table}[h]
\centering
\caption{Hyperparameter settings.}
\label{tab:meth-hyperparams}
\begin{tabular}{@{}llc@{}}
\toprule
Parameter & Description & Value \\
\midrule
$\eta$ & Learning rate & 0.001 \\
$\lambda$ & Regularization & 0.01 \\
$T$ & Max iterations & 1000 \\
Batch size & Training batch & 64 \\
\bottomrule
\end{tabular}
\end{table}

\section{Summary}
\label{sec:meth-summary}

This chapter has presented the proposed methodology, including:
\begin{itemize}
    \item Data preprocessing procedures
    \item The core algorithm with theoretical guarantees
    \item Implementation details for reproducibility
\end{itemize}

The next chapter presents the experimental evaluation of this methodology.
