% Scientific Lab Report Template
% For undergraduate and graduate laboratory courses
%
% Compile with: pdflatex main.tex
%               bibtex main
%               pdflatex main.tex
%               pdflatex main.tex

\documentclass[12pt,a4paper]{article}

% Essential packages
\usepackage[margin=1in]{geometry}
\usepackage{times}
\usepackage{graphicx}
\usepackage{booktabs}
\usepackage{amsmath}
\usepackage{amssymb}
\usepackage{siunitx}
\usepackage{float}
\usepackage{caption}
\usepackage{subcaption}
\usepackage{hyperref}
\usepackage{natbib}
\usepackage{setspace}
\usepackage{fancyhdr}
\usepackage{lastpage}

% Configure siunitx
\sisetup{
    separate-uncertainty = true,
    multi-part-units = single,
}

% Page style
\pagestyle{fancy}
\fancyhf{}
\rhead{{{COURSE_CODE}} - Lab Report}
\lhead{{{AUTHOR}}}
\rfoot{Page \thepage\ of \pageref{LastPage}}
\renewcommand{\headrulewidth}{0.4pt}
\renewcommand{\footrulewidth}{0.4pt}

% Line spacing
\onehalfspacing

% Title page information
\title{
    \textbf{{{TITLE}}} \\
    \vspace{0.5cm}
    \large {{COURSE_CODE}}: {{COURSE_NAME}} \\
    Lab Section: {{LAB_SECTION}}
}

\author{
    {{AUTHOR}} \\
    Student ID: {{STUDENT_ID}} \\
    Lab Partner(s): {{LAB_PARTNERS}} \\
    \\
    Instructor: {{INSTRUCTOR}} \\
    Teaching Assistant: {{TA_NAME}}
}

\date{
    Experiment Performed: {{EXPERIMENT_DATE}} \\
    Report Submitted: {{SUBMISSION_DATE}}
}

\begin{document}

% Title page
\maketitle
\thispagestyle{empty}
\newpage

% Abstract
\begin{abstract}
{{ABSTRACT}}
\end{abstract}

\section{Introduction}

\subsection{Background}

The purpose of this experiment is to investigate {{EXPERIMENT_PURPOSE}}. Understanding this phenomenon is important because {{IMPORTANCE_STATEMENT}}.

The theoretical background for this experiment is based on the principle that {{THEORY_DESCRIPTION}}. According to established scientific principles \citep{serway2018physics}, the relationship can be expressed as:

\begin{equation}
    {{EQUATION_VARIABLE}} = {{EQUATION_EXPRESSION}}
    \label{eq:main}
\end{equation}

where:
\begin{itemize}
    \item ${{VARIABLE_1}}$ = {{VARIABLE_1_DESCRIPTION}} [{{VARIABLE_1_UNITS}}]
    \item ${{VARIABLE_2}}$ = {{VARIABLE_2_DESCRIPTION}} [{{VARIABLE_2_UNITS}}]
    \item ${{VARIABLE_3}}$ = {{VARIABLE_3_DESCRIPTION}} [{{VARIABLE_3_UNITS}}]
\end{itemize}

\subsection{Objectives}

The specific objectives of this experiment are:
\begin{enumerate}
    \item To measure {{OBJECTIVE_1}}
    \item To determine the relationship between {{OBJECTIVE_2}}
    \item To verify the theoretical predictions for {{OBJECTIVE_3}}
    \item To develop skills in {{OBJECTIVE_4}}
\end{enumerate}

\subsection{Hypothesis}

Based on the theoretical background, we hypothesize that {{HYPOTHESIS}}.

\section{Materials and Methods}

\subsection{Materials}

The following equipment and materials were used:
\begin{itemize}
    \item {{EQUIPMENT_1}} ({{EQUIPMENT_1_SPECS}})
    \item {{EQUIPMENT_2}} ({{EQUIPMENT_2_SPECS}})
    \item {{EQUIPMENT_3}} ({{EQUIPMENT_3_SPECS}})
    \item {{MATERIAL_1}}
    \item {{MATERIAL_2}}
    \item Stopwatch (resolution: \SI{0.01}{\second})
    \item Digital balance (precision: \SI{0.01}{\gram})
    \item Meter stick (precision: \SI{1}{\milli\meter})
\end{itemize}

\subsection{Experimental Setup}

Figure~\ref{fig:setup} shows the experimental apparatus used in this investigation.

\begin{figure}[H]
    \centering
    % \includegraphics[width=0.7\textwidth]{figures/setup.png}
    \fbox{\parbox{0.7\textwidth}{\centering\vspace{3cm}[Diagram of experimental setup]\vspace{3cm}}}
    \caption{Schematic diagram of the experimental setup showing {{SETUP_DESCRIPTION}}.}
    \label{fig:setup}
\end{figure}

\subsection{Procedure}

The experiment was conducted according to the following procedure:

\begin{enumerate}
    \item The apparatus was assembled as shown in Figure~\ref{fig:setup}.
    \item {{PROCEDURE_STEP_1}}
    \item {{PROCEDURE_STEP_2}}
    \item {{PROCEDURE_STEP_3}}
    \item {{PROCEDURE_STEP_4}}
    \item Each measurement was repeated three times to ensure reproducibility.
    \item All data were recorded in the laboratory notebook.
\end{enumerate}

\subsection{Safety Considerations}

The following safety precautions were observed:
\begin{itemize}
    \item {{SAFETY_1}}
    \item {{SAFETY_2}}
    \item Appropriate personal protective equipment was worn throughout the experiment.
\end{itemize}

\section{Results}

\subsection{Raw Data}

Table~\ref{tab:raw_data} presents the raw data collected during the experiment.

\begin{table}[H]
    \centering
    \caption{Raw experimental data collected during the investigation.}
    \label{tab:raw_data}
    \begin{tabular}{cccc}
        \toprule
        \textbf{Trial} & \textbf{{{COLUMN_1}} ({{UNITS_1}})} & \textbf{{{COLUMN_2}} ({{UNITS_2}})} & \textbf{{{COLUMN_3}} ({{UNITS_3}})} \\
        \midrule
        1 & {{DATA_1_1}} & {{DATA_1_2}} & {{DATA_1_3}} \\
        2 & {{DATA_2_1}} & {{DATA_2_2}} & {{DATA_2_3}} \\
        3 & {{DATA_3_1}} & {{DATA_3_2}} & {{DATA_3_3}} \\
        4 & {{DATA_4_1}} & {{DATA_4_2}} & {{DATA_4_3}} \\
        5 & {{DATA_5_1}} & {{DATA_5_2}} & {{DATA_5_3}} \\
        \midrule
        Mean & {{MEAN_1}} & {{MEAN_2}} & {{MEAN_3}} \\
        Std Dev & {{STD_1}} & {{STD_2}} & {{STD_3}} \\
        \bottomrule
    \end{tabular}
\end{table}

\subsection{Calculated Results}

Using equation~\eqref{eq:main} and the measured values, the following calculations were performed:

\begin{equation}
    {{CALCULATED_RESULT}} = {{CALCULATION_EXPRESSION}} = {{FINAL_VALUE}} \pm {{UNCERTAINTY}} \text{ {{RESULT_UNITS}}}
\end{equation}

The percent error compared to the accepted value is:

\begin{equation}
    \text{Percent Error} = \frac{|{{EXPERIMENTAL_VALUE}} - {{ACCEPTED_VALUE}}|}{{{ACCEPTED_VALUE}}} \times 100\% = {{PERCENT_ERROR}}\%
\end{equation}

\subsection{Graphical Analysis}

Figure~\ref{fig:graph} shows the relationship between {{X_VARIABLE}} and {{Y_VARIABLE}}.

\begin{figure}[H]
    \centering
    % \includegraphics[width=0.8\textwidth]{figures/graph.png}
    \fbox{\parbox{0.8\textwidth}{\centering\vspace{5cm}[Graph showing {{X_VARIABLE}} vs {{Y_VARIABLE}}]\vspace{5cm}}}
    \caption{Plot of {{Y_VARIABLE}} as a function of {{X_VARIABLE}}. The linear fit yields a slope of {{SLOPE}} $\pm$ {{SLOPE_ERROR}} and $R^2$ = {{R_SQUARED}}.}
    \label{fig:graph}
\end{figure}

The linear regression analysis yields:
\begin{itemize}
    \item Slope: {{SLOPE}} $\pm$ {{SLOPE_ERROR}} {{SLOPE_UNITS}}
    \item Y-intercept: {{INTERCEPT}} $\pm$ {{INTERCEPT_ERROR}} {{INTERCEPT_UNITS}}
    \item Correlation coefficient ($R^2$): {{R_SQUARED}}
\end{itemize}

\section{Discussion}

\subsection{Analysis of Results}

The experimental results {{SUPPORT_OR_REFUTE}} the initial hypothesis. The measured value of {{MEASURED_QUANTITY}} (\SI{{{MEASURED_VALUE}}}{{MEASURED_UNITS}}}) compares {{FAVORABLY_OR_UNFAVORABLY}} with the theoretical prediction (\SI{{{THEORETICAL_VALUE}}}{{THEORETICAL_UNITS}}}).

The percent error of {{PERCENT_ERROR}}\% indicates {{ERROR_INTERPRETATION}}. Several factors may contribute to this discrepancy:

\begin{enumerate}
    \item \textbf{Systematic errors:} {{SYSTEMATIC_ERROR_DISCUSSION}}
    \item \textbf{Random errors:} {{RANDOM_ERROR_DISCUSSION}}
    \item \textbf{Instrumental limitations:} {{INSTRUMENTAL_ERROR_DISCUSSION}}
\end{enumerate}

\subsection{Sources of Error}

The main sources of uncertainty in this experiment include:

\begin{itemize}
    \item Measurement uncertainty in {{ERROR_SOURCE_1}} ($\pm$ {{UNCERTAINTY_1}})
    \item {{ERROR_SOURCE_2}}
    \item Environmental factors such as {{ENVIRONMENTAL_FACTOR}}
\end{itemize}

\subsection{Comparison with Theory}

The relationship observed in Figure~\ref{fig:graph} demonstrates {{RELATIONSHIP_DESCRIPTION}}. This is consistent with the theoretical model described in the introduction, which predicts a {{PREDICTED_RELATIONSHIP}} relationship.

\subsection{Improvements}

Future experiments could be improved by:
\begin{enumerate}
    \item {{IMPROVEMENT_1}}
    \item {{IMPROVEMENT_2}}
    \item Using more precise measuring instruments
    \item Increasing the number of trials
\end{enumerate}

\section{Conclusion}

This experiment successfully {{CONCLUSION_SUMMARY}}. The key findings include:

\begin{enumerate}
    \item {{FINDING_1}}
    \item {{FINDING_2}}
    \item The experimental results agree with theoretical predictions within {{AGREEMENT_PERCENT}}\%.
\end{enumerate}

The hypothesis was {{CONFIRMED_OR_REJECTED}} based on the experimental evidence. The skills developed during this experiment include {{SKILLS_DEVELOPED}}.

\section*{Acknowledgments}

{{ACKNOWLEDGMENTS}}

\newpage
\bibliography{references}
\bibliographystyle{plain}

\newpage
\appendix
\section{Sample Calculations}
\label{app:calculations}

\textbf{Example calculation for {{CALCULATED_QUANTITY}}:}

Given:
\begin{itemize}
    \item ${{GIVEN_VAR_1}}$ = {{GIVEN_VALUE_1}} {{GIVEN_UNITS_1}}
    \item ${{GIVEN_VAR_2}}$ = {{GIVEN_VALUE_2}} {{GIVEN_UNITS_2}}
\end{itemize}

Calculation:
\begin{align}
    {{CALC_VARIABLE}} &= {{CALC_FORMULA}} \\
    &= {{CALC_SUBSTITUTION}} \\
    &= {{CALC_RESULT}} \text{ {{CALC_UNITS}}}
\end{align}

\section{Raw Data Tables}
\label{app:rawdata}

[Include original data sheets or additional data tables here]

\end{document}
