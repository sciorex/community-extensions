% Chapter 2: Literature Review / Background

\chapter{Literature Review}
\label{ch:literature}

This chapter provides a comprehensive review of the literature relevant to this dissertation. We begin with foundational concepts, then survey existing approaches, and conclude by identifying gaps that motivate our research.

\section{Theoretical Foundations}
\label{sec:foundations}

Before discussing specific methods and approaches, we establish the theoretical foundations necessary for understanding the contributions of this dissertation.

\subsection{Fundamental Concepts}
\label{subsec:concepts}

\begin{definition}[Key Concept]
\label{def:key-concept}
A \textit{key concept} is defined as [formal definition]. Mathematically, we write:
\begin{equation}
    f(x) = \int_{a}^{b} g(x,y) \, dy
    \label{eq:definition}
\end{equation}
\end{definition}

The significance of this definition lies in its ability to [explanation of significance].

\begin{theorem}[Important Result]
\label{thm:important}
Under conditions [conditions], the following holds:
\begin{equation}
    \lim_{n \to \infty} P(X_n = x) = p(x)
\end{equation}
\end{theorem}

\begin{proof}
The proof proceeds in three steps. First, [step 1]. Second, [step 2]. Finally, [step 3]. This completes the proof.
\end{proof}

\subsection{Mathematical Framework}
\label{subsec:framework}

The mathematical framework underlying our approach builds on [area]. Let $\mathcal{X}$ denote the input space and $\mathcal{Y}$ the output space. We consider the problem of finding a mapping $f: \mathcal{X} \to \mathcal{Y}$ that minimizes:

\begin{equation}
    \mathcal{L}(f) = \mathbb{E}_{(x,y) \sim P} \left[ \ell(f(x), y) \right] + \lambda \Omega(f)
    \label{eq:objective}
\end{equation}

where $\ell$ is a loss function, $\Omega$ is a regularizer, and $\lambda > 0$ controls the trade-off.

\section{Area One: [Topic]}
\label{sec:area-one}

This section reviews research in [topic area], which forms the foundation for [aspect of our work].

\subsection{Historical Development}
\label{subsec:history}

The study of [topic] began with the seminal work of [Author] \cite{ref1}, who introduced [key contribution]. This work established [foundational result].

Subsequent research extended these ideas in several directions:

\begin{itemize}
    \item [Author et al.] \cite{ref2} proposed [contribution], demonstrating [result].
    \item [Author et al.] \cite{ref3} developed [approach], achieving [outcome].
    \item [Author et al.] \cite{ref4} investigated [aspect], finding [conclusion].
\end{itemize}

\subsection{Current State of the Art}
\label{subsec:sota}

Recent advances have significantly improved the state of the art. Table~\ref{tab:comparison} summarizes key methods and their characteristics.

\begin{table}[t]
\centering
\caption{Comparison of existing methods in [area].}
\label{tab:comparison}
\begin{tabular}{lccc}
\toprule
Method & Approach & Advantages & Limitations \\
\midrule
Method A & [Type] & [Pros] & [Cons] \\
Method B & [Type] & [Pros] & [Cons] \\
Method C & [Type] & [Pros] & [Cons] \\
Method D & [Type] & [Pros] & [Cons] \\
\bottomrule
\end{tabular}
\end{table}

\section{Area Two: [Topic]}
\label{sec:area-two}

This section examines [second major area], which relates to [connection to dissertation].

\subsection{Key Approaches}
\label{subsec:approaches}

Several distinct approaches have been developed for [problem]:

\paragraph{Approach Type 1.}
The first class of methods [description]. These approaches are characterized by [features]. Notable examples include [methods] \cite{ref5,ref6}.

\paragraph{Approach Type 2.}
An alternative paradigm involves [description]. This approach differs from the previous in that [key differences]. Representative works include [methods] \cite{ref7,ref8}.

\paragraph{Approach Type 3.}
More recently, researchers have explored [description]. These methods offer [advantages] but suffer from [limitations].

\subsection{Comparative Analysis}
\label{subsec:comparative}

Comparing these approaches reveals several insights:

\begin{enumerate}
    \item \textbf{Finding 1:} [Description of comparative finding].
    \item \textbf{Finding 2:} [Description of comparative finding].
    \item \textbf{Finding 3:} [Description of comparative finding].
\end{enumerate}

\section{Area Three: [Topic]}
\label{sec:area-three}

[Third major area of related work, following similar structure as above.]

\subsection{Relevant Methods}
[Content]

\subsection{Analysis}
[Content]

\section{Research Gaps and Opportunities}
\label{sec:gaps}

Based on our review of the literature, we identify the following gaps that motivate the research in this dissertation:

\begin{description}
    \item[Gap 1:] Existing methods fail to [description of gap]. This gap is addressed in Chapter~[X] through [our approach].

    \item[Gap 2:] Current approaches do not consider [description of gap]. We address this in Chapter~[X] by [our contribution].

    \item[Gap 3:] There is limited understanding of [description of gap]. Chapter~[X] provides [our contribution].
\end{description}

Table~\ref{tab:gaps} summarizes how our contributions address these gaps.

\begin{table}[t]
\centering
\caption{Summary of research gaps and corresponding contributions.}
\label{tab:gaps}
\begin{tabular}{lll}
\toprule
Gap & Our Contribution & Chapter \\
\midrule
Gap 1 & Contribution 1 & Chapter~X \\
Gap 2 & Contribution 2 & Chapter~X \\
Gap 3 & Contribution 3 & Chapter~X \\
\bottomrule
\end{tabular}
\end{table}

\section{Summary}
\label{sec:ch2-summary}

This chapter has reviewed the literature in [areas] relevant to this dissertation. We have:

\begin{enumerate}
    \item Established the theoretical foundations in Section~\ref{sec:foundations}.
    \item Surveyed existing methods in Sections~\ref{sec:area-one}--\ref{sec:area-three}.
    \item Identified key research gaps in Section~\ref{sec:gaps}.
\end{enumerate}

The next chapter presents [content of next chapter], our approach to addressing Gap~1.
