% Appendix A: Supplementary Material

\chapter{Supplementary Material}
\label{app:supplementary}

This appendix provides supplementary material to support the main text of the dissertation.

\section{Mathematical Proofs}
\label{app:proofs}

This section contains detailed proofs of theorems stated in the main text.

\subsection{Proof of Theorem~\ref{thm:important}}
\label{app:proof-theorem}

We provide a detailed proof of Theorem~\ref{thm:important} from Chapter~\ref{ch:literature}.

\begin{proof}
[Detailed proof steps]

Let us first establish the following lemma:

\begin{lemma}
\label{lem:helper}
Under the conditions of Theorem~\ref{thm:important}, [statement of lemma].
\end{lemma}

\begin{proof}[Proof of Lemma~\ref{lem:helper}]
[Proof of the lemma]
\end{proof}

Using Lemma~\ref{lem:helper}, we can now prove the main theorem.

[Remainder of proof]

This completes the proof.
\end{proof}

\section{Additional Experimental Results}
\label{app:experiments}

This section presents additional experimental results that supplement the findings in Chapter~[X].

\subsection{Dataset Statistics}
\label{app:datasets}

Table~\ref{tab:dataset-stats} provides detailed statistics for all datasets used in our experiments.

\begin{longtable}{lccccc}
\caption{Detailed dataset statistics.}
\label{tab:dataset-stats}\\
\toprule
Dataset & Train & Val & Test & Classes & Features \\
\midrule
\endfirsthead
\caption[]{(continued)}\\
\toprule
Dataset & Train & Val & Test & Classes & Features \\
\midrule
\endhead
\bottomrule
\endfoot
Dataset A & 50,000 & 10,000 & 10,000 & 10 & 784 \\
Dataset B & 100,000 & 20,000 & 20,000 & 100 & 1024 \\
Dataset C & 200,000 & 25,000 & 25,000 & 1000 & 2048 \\
\end{longtable}

\subsection{Hyperparameter Settings}
\label{app:hyperparameters}

Table~\ref{tab:hyperparams} lists the hyperparameter settings used for each experiment.

\begin{table}[h]
\centering
\caption{Hyperparameter settings for all experiments.}
\label{tab:hyperparams}
\begin{tabular}{lcccc}
\toprule
Hyperparameter & Exp. 1 & Exp. 2 & Exp. 3 & Exp. 4 \\
\midrule
Learning rate & 0.001 & 0.0001 & 0.001 & 0.0005 \\
Batch size & 64 & 128 & 64 & 256 \\
Epochs & 100 & 200 & 150 & 300 \\
Weight decay & 0.01 & 0.01 & 0.001 & 0.0001 \\
Dropout & 0.1 & 0.2 & 0.1 & 0.3 \\
\bottomrule
\end{tabular}
\end{table}

\subsection{Additional Ablation Studies}
\label{app:ablations}

[Additional ablation study results]

\section{Implementation Details}
\label{app:implementation}

This section provides implementation details to facilitate reproducibility.

\subsection{System Configuration}
\label{app:system}

All experiments were conducted on the following hardware and software configuration:

\begin{itemize}
    \item \textbf{Hardware:}
    \begin{itemize}
        \item CPU: [Specification]
        \item GPU: [Specification]
        \item Memory: [Specification]
    \end{itemize}

    \item \textbf{Software:}
    \begin{itemize}
        \item Operating System: [Version]
        \item Python: [Version]
        \item PyTorch: [Version]
        \item CUDA: [Version]
    \end{itemize}
\end{itemize}

\subsection{Code Availability}
\label{app:code}

The source code for all experiments is available at:
\begin{center}
\url{https://github.com/[username]/[repository]}
\end{center}

The repository includes:
\begin{itemize}
    \item Implementation of all proposed methods
    \item Scripts for data preprocessing
    \item Configuration files for reproducing experiments
    \item Pre-trained model weights
\end{itemize}

\section{Survey Instruments}
\label{app:survey}

[If applicable, include survey questions or other instruments used in research]

\section{Glossary}
\label{app:glossary}

\begin{description}
    \item[Term 1:] Definition of the first term.
    \item[Term 2:] Definition of the second term.
    \item[Term 3:] Definition of the third term.
\end{description}
