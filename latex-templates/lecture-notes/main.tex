%% Lecture Notes Template
%% Academic lecture notes with theorems, definitions, and examples
%%
%% Placeholders use {{PLACEHOLDER}} syntax for easy substitution

\documentclass[11pt,a4paper]{article}

% Packages
\usepackage[utf8]{inputenc}
\usepackage[T1]{fontenc}
\usepackage{lmodern}
\usepackage[margin=1in]{geometry}
\usepackage{amsmath,amssymb,amsthm}
\usepackage{tcolorbox}
\usepackage{enumitem}
\usepackage{hyperref}
\usepackage{xcolor}
\usepackage{fancyhdr}
\usepackage{titlesec}
\usepackage{graphicx}
\usepackage{algorithm}
\usepackage{algpseudocode}
\usepackage{booktabs}

% Colors
\definecolor{theoremcolor}{RGB}{0,51,102}
\definecolor{definitioncolor}{RGB}{102,51,0}
\definecolor{examplecolor}{RGB}{0,102,51}
\definecolor{remarkcolor}{RGB}{102,102,102}
\definecolor{warningcolor}{RGB}{153,51,0}

% Hyperref setup
\hypersetup{
    colorlinks=true,
    linkcolor=theoremcolor,
    urlcolor=theoremcolor,
    citecolor=theoremcolor
}

% Theorem environments
\tcbuselibrary{theorems,skins,breakable}

\newtcbtheorem[number within=section]{theorem}{Theorem}{
    colback=theoremcolor!5,
    colframe=theoremcolor!80,
    fonttitle=\bfseries,
    breakable
}{thm}

\newtcbtheorem[number within=section]{definition}{Definition}{
    colback=definitioncolor!5,
    colframe=definitioncolor!80,
    fonttitle=\bfseries,
    breakable
}{def}

\newtcbtheorem[number within=section]{lemma}{Lemma}{
    colback=theoremcolor!5,
    colframe=theoremcolor!60,
    fonttitle=\bfseries,
    breakable
}{lem}

\newtcbtheorem[number within=section]{corollary}{Corollary}{
    colback=theoremcolor!5,
    colframe=theoremcolor!60,
    fonttitle=\bfseries,
    breakable
}{cor}

\newtcbtheorem[number within=section]{proposition}{Proposition}{
    colback=theoremcolor!5,
    colframe=theoremcolor!70,
    fonttitle=\bfseries,
    breakable
}{prop}

% Example environment
\newtcbtheorem[number within=section]{example}{Example}{
    colback=examplecolor!5,
    colframe=examplecolor!80,
    fonttitle=\bfseries,
    breakable
}{ex}

% Remark environment
\newtcbtheorem[number within=section]{remark}{Remark}{
    colback=remarkcolor!5,
    colframe=remarkcolor!60,
    fonttitle=\bfseries,
    breakable
}{rem}

% Warning/Important note environment
\newtcolorbox{warning}{
    colback=warningcolor!5,
    colframe=warningcolor!80,
    title={\textbf{Important}},
    breakable
}

% Exercise environment
\newcounter{exercise}[section]
\renewcommand{\theexercise}{\thesection.\arabic{exercise}}
\newenvironment{exercise}[1][]{
    \refstepcounter{exercise}
    \par\medskip\noindent
    \textbf{Exercise \theexercise#1.}
    \rmfamily
}{\medskip}

% Solution environment (can be hidden)
\newenvironment{solution}{
    \par\medskip\noindent
    \textit{Solution.}
}{\hfill$\square$\medskip}

% Page style
\pagestyle{fancy}
\fancyhf{}
\fancyhead[L]{{{COURSE_CODE}}: {{COURSE_NAME}}}
\fancyhead[R]{Lecture {{LECTURE_NUM}}}
\fancyfoot[C]{\thepage}
\renewcommand{\headrulewidth}{0.4pt}

% Title
\title{
    \vspace{-1cm}
    {\large {{COURSE_CODE}}: {{COURSE_NAME}}}\\[0.5em]
    {\LARGE\bfseries Lecture {{LECTURE_NUM}}: {{LECTURE_TITLE}}}
}
\author{{{INSTRUCTOR}}\\{\small {{INSTITUTION}}}}
\date{{{DATE}}}

\begin{document}

\maketitle

\tableofcontents
\vspace{1em}

% Learning objectives
\begin{tcolorbox}[colback=blue!5,colframe=blue!40,title={\textbf{Learning Objectives}}]
By the end of this lecture, you will be able to:
\begin{itemize}[nosep]
    \item {{OBJECTIVE_1}}
    \item {{OBJECTIVE_2}}
    \item {{OBJECTIVE_3}}
\end{itemize}
\end{tcolorbox}

%% =============================================================================
\section{Introduction}
%% =============================================================================

{{INTRO_TEXT}}

In this lecture, we will explore the fundamental concepts of {{TOPIC}}. Building upon our previous discussion of {{PREVIOUS_TOPIC}}, we will develop a deeper understanding of {{KEY_CONCEPT}}.

%% =============================================================================
\section{Fundamental Definitions}
%% =============================================================================

We begin by establishing the key definitions that will be used throughout this lecture.

\begin{definition}{{{DEFINITION_1_NAME}}}{def1}
{{DEFINITION_1_TEXT}}
\end{definition}

Let us consider what this definition implies in practice.

\begin{example}{Illustrating {{DEFINITION_1_NAME}}}{ex1}
{{EXAMPLE_1_TEXT}}

Consider the case where $x = {{EXAMPLE_VALUE}}$. Then we have:
\[
    {{EXAMPLE_EQUATION}}
\]
\end{example}

\begin{definition}{{{DEFINITION_2_NAME}}}{def2}
{{DEFINITION_2_TEXT}}
\end{definition}

\begin{remark}{Relationship between definitions}{rem1}
Note that {{DEFINITION_1_NAME}} and {{DEFINITION_2_NAME}} are related through {{RELATIONSHIP_TEXT}}.
\end{remark}

%% =============================================================================
\section{Main Results}
%% =============================================================================

We now present the main theoretical results of this lecture.

\begin{theorem}{{{THEOREM_NAME}}}{main}
{{THEOREM_STATEMENT}}
\end{theorem}

\begin{proof}
{{PROOF_TEXT}}

We proceed by {{PROOF_METHOD}}.

\textit{Step 1:} {{PROOF_STEP_1}}

\textit{Step 2:} {{PROOF_STEP_2}}

Therefore, we conclude that {{PROOF_CONCLUSION}}.
\end{proof}

\begin{corollary}{Consequence of {{THEOREM_NAME}}}{cor1}
{{COROLLARY_TEXT}}
\end{corollary}

\begin{example}{Application of {{THEOREM_NAME}}}{ex2}
{{APPLICATION_EXAMPLE}}
\end{example}

%% =============================================================================
\section{Extended Discussion}
%% =============================================================================

\subsection{Special Cases}

{{SPECIAL_CASES_TEXT}}

\begin{lemma}{{{LEMMA_NAME}}}{lem1}
{{LEMMA_STATEMENT}}
\end{lemma}

\subsection{Connections to Other Topics}

{{CONNECTIONS_TEXT}}

\begin{warning}
{{WARNING_TEXT}}
\end{warning}

%% =============================================================================
\section{Worked Examples}
%% =============================================================================

\begin{example}{{{WORKED_EXAMPLE_NAME}}}{worked}
\textbf{Problem:} {{WORKED_PROBLEM}}

\textbf{Solution:}
{{WORKED_SOLUTION}}
\end{example}

%% =============================================================================
\section{Summary}
%% =============================================================================

In this lecture, we covered:
\begin{itemize}
    \item {{SUMMARY_POINT_1}}
    \item {{SUMMARY_POINT_2}}
    \item {{SUMMARY_POINT_3}}
\end{itemize}

\textbf{Key takeaways:}
\begin{enumerate}
    \item {{TAKEAWAY_1}}
    \item {{TAKEAWAY_2}}
\end{enumerate}

%% =============================================================================
\section{Exercises}
%% =============================================================================

\begin{exercise}
{{EXERCISE_1}}
\end{exercise}

\begin{exercise}
{{EXERCISE_2}}
\end{exercise}

\begin{exercise}[*]
\textbf{(Challenge)} {{EXERCISE_3}}
\end{exercise}

%% =============================================================================
\section*{Further Reading}
%% =============================================================================

For additional material on the topics covered in this lecture, see:
\begin{itemize}
    \item {{READING_1}}
    \item {{READING_2}}
\end{itemize}

\textbf{Next lecture:} {{NEXT_LECTURE_TOPIC}}

%% Bibliography (optional - uncomment if using citations)
% \bibliographystyle{plain}
% \bibliography{references}

\end{document}
