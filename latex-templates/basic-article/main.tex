\documentclass[11pt,a4paper]{article}

\usepackage[utf8]{inputenc}
\usepackage[T1]{fontenc}
\usepackage{amsmath}
\usepackage{amssymb}
\usepackage{graphicx}
\usepackage{hyperref}
\usepackage{geometry}
\usepackage{fancyhdr}

\geometry{margin=1in}

\pagestyle{fancy}
\fancyhf{}
\rhead{\thepage}
\lhead{\leftmark}

\title{{{TITLE}}}
\author{{{AUTHOR}}\\
  \texttt{{{EMAIL}}}}
\date{\today}

\begin{document}

\maketitle

\begin{abstract}
This document is a basic article template. Replace this text with your abstract. The abstract should provide a concise summary of your document's content and purpose.
\end{abstract}

\tableofcontents
\newpage

\section{Introduction}
This is the introduction section. Replace this text with your introduction content.

\section{Background}
Provide background information here. You can cite references like this~\cite{example2024}.

\section{Main Content}
This is the main content section. Organize your material as needed.

\subsection{Subsection Example}
You can use subsections to organize your content.

\subsubsection{Subsubsection Example}
Even deeper organization if needed.

\section{Figures and Tables}

You can include figures:

\begin{figure}[h]
\centering
% \includegraphics[width=0.5\textwidth]{your-figure.pdf}
\caption{Example figure caption}
\label{fig:example}
\end{figure}

And tables:

\begin{table}[h]
\centering
\begin{tabular}{lcc}
\hline
Item & Value 1 & Value 2 \\
\hline
A & 1.0 & 2.0 \\
B & 3.0 & 4.0 \\
C & 5.0 & 6.0 \\
\hline
\end{tabular}
\caption{Example table caption}
\label{tab:example}
\end{table}

\section{Conclusion}
Summarize your content here.

\bibliographystyle{plain}
\bibliography{references}

\end{document}
