%% Beamer Presentation with Metropolis Theme
%% Modern, minimal presentation template
%%
%% Compile with: xelatex main.tex (recommended) or pdflatex main.tex

\documentclass[aspectratio=169]{beamer}

%% Metropolis theme
\usetheme{metropolis}

%% Packages
\usepackage{appendixnumberbeamer}
\usepackage{booktabs}
\usepackage{graphicx}
\usepackage{amsmath}
\usepackage{amssymb}
\usepackage[scale=2]{ccicons}

%% For XeLaTeX (recommended), uncomment the next line:
% \usepackage{fontspec}
%% If using pdflatex (default):
\usepackage[T1]{fontenc}
\usepackage{lmodern}

%% Bibliography (optional)
\usepackage[backend=biber,style=authoryear]{biblatex}
\addbibresource{references.bib}

%% Custom colors (optional)
% \setbeamercolor{frametitle}{bg=mDarkTeal}
% \setbeamercolor{progress bar}{fg=mDarkTeal}

%% Presentation metadata
\title{{{TITLE}}}
\subtitle{{{SUBTITLE}}}
\author{{{AUTHOR}}}
\institute{{{INSTITUTE}}}
\date{{{DATE}}}

%% Logo (optional)
% \titlegraphic{\hfill\includegraphics[height=1.5cm]{logo.pdf}}

\begin{document}

%% ===================
%% TITLE SLIDE
%% ===================
\maketitle

%% ===================
%% TABLE OF CONTENTS
%% ===================
\begin{frame}{Outline}
    \setbeamertemplate{section in toc}[sections numbered]
    \tableofcontents[hideallsubsections]
\end{frame}

%% ===================
%% SECTION: INTRODUCTION
%% ===================
\section{Introduction}

\begin{frame}{Background}
    \textbf{Context and Motivation}

    \begin{itemize}
        \item Problem area has significant real-world impact
        \item Current approaches have notable limitations
        \item Opportunity for novel contributions
    \end{itemize}

    \pause

    \textbf{Key Challenge}

    How can we address limitation X while maintaining efficiency?
\end{frame}

\begin{frame}{Research Questions}
    \begin{enumerate}
        \item What is the relationship between X and Y?
        \item How can we improve upon existing methods?
        \item What are the practical implications?
    \end{enumerate}

    \pause

    \alert{Main Contribution:} A novel framework that achieves state-of-the-art results.
\end{frame}

%% ===================
%% SECTION: METHODS
%% ===================
\section{Methods}

\begin{frame}{Approach Overview}
    \begin{columns}[T]
        \begin{column}{0.5\textwidth}
            \textbf{Key Components}
            \begin{itemize}
                \item Component A
                \item Component B
                \item Component C
            \end{itemize}
        \end{column}
        \begin{column}{0.5\textwidth}
            \textbf{Innovations}
            \begin{itemize}
                \item Novel technique 1
                \item Novel technique 2
                \item Integration strategy
            \end{itemize}
        \end{column}
    \end{columns}
\end{frame}

\begin{frame}{Mathematical Framework}
    The core optimization problem:

    \begin{equation*}
        \min_{\theta} \mathcal{L}(\theta) = \sum_{i=1}^{N} \ell(f_\theta(x_i), y_i) + \lambda \|\theta\|_2^2
    \end{equation*}

    \pause

    \begin{block}{Key Insight}
        By reformulating the problem, we achieve computational efficiency while maintaining accuracy.
    \end{block}
\end{frame}

\begin{frame}{Algorithm}
    \begin{algorithm}[H]
        \textbf{Input:} Data $\mathcal{D}$, parameters $\theta_0$, learning rate $\eta$ \\
        \textbf{Output:} Optimized parameters $\theta^*$

        \begin{enumerate}
            \item Initialize $\theta \leftarrow \theta_0$
            \item \textbf{for} $t = 1, \ldots, T$ \textbf{do}
            \item \quad Compute gradient $g_t \leftarrow \nabla_\theta \mathcal{L}(\theta)$
            \item \quad Update $\theta \leftarrow \theta - \eta \cdot g_t$
            \item \textbf{end for}
            \item \textbf{return} $\theta$
        \end{enumerate}
    \end{algorithm}
\end{frame}

%% ===================
%% SECTION: RESULTS
%% ===================
\section{Results}

\begin{frame}{Experimental Setup}
    \textbf{Datasets}
    \begin{itemize}
        \item Dataset A: 10,000 samples, domain X
        \item Dataset B: 50,000 samples, domain Y
        \item Dataset C: 100,000 samples, domain Z
    \end{itemize}

    \textbf{Baselines}
    \begin{itemize}
        \item Method 1 \parencite{smith2023method}
        \item Method 2 \parencite{johnson2024approach}
        \item Method 3 (state-of-the-art)
    \end{itemize}
\end{frame}

\begin{frame}{Main Results}
    \begin{table}
        \centering
        \begin{tabular}{@{}lccc@{}}
            \toprule
            \textbf{Method} & \textbf{Accuracy} & \textbf{F1 Score} & \textbf{Time} \\
            \midrule
            Baseline 1 & 78.2\% & 0.76 & 45s \\
            Baseline 2 & 82.4\% & 0.81 & 62s \\
            Previous SOTA & 85.1\% & 0.84 & 58s \\
            \textbf{Ours} & \textbf{89.7\%} & \textbf{0.88} & \textbf{32s} \\
            \bottomrule
        \end{tabular}
        \caption{Performance comparison on Dataset A}
    \end{table}
\end{frame}

\begin{frame}{Visual Results}
    \begin{figure}
        \centering
        %% Replace with actual figure
        % \includegraphics[width=0.8\textwidth]{results_figure.pdf}
        \fbox{\parbox{0.7\textwidth}{\centering\vspace{2cm}Results Figure\vspace{2cm}}}
        \caption{Performance across different conditions}
    \end{figure}
\end{frame}

\begin{frame}{Ablation Study}
    \begin{columns}[T]
        \begin{column}{0.5\textwidth}
            \textbf{Component Analysis}
            \begin{table}
                \small
                \begin{tabular}{@{}lc@{}}
                    \toprule
                    Configuration & Acc. \\
                    \midrule
                    Full model & 89.7\% \\
                    w/o Component A & 86.2\% \\
                    w/o Component B & 84.8\% \\
                    w/o Component C & 87.1\% \\
                    \bottomrule
                \end{tabular}
            \end{table}
        \end{column}
        \begin{column}{0.5\textwidth}
            \textbf{Key Observations}
            \begin{itemize}
                \item Component B most critical
                \item Synergy between A and C
                \item Each component contributes
            \end{itemize}
        \end{column}
    \end{columns}
\end{frame}

%% ===================
%% SECTION: CONCLUSION
%% ===================
\section{Conclusion}

\begin{frame}{Summary}
    \textbf{Contributions}
    \begin{enumerate}
        \item Novel framework for problem X
        \item Theoretical analysis with guarantees
        \item Extensive experimental validation
        \item Open-source implementation
    \end{enumerate}

    \pause

    \textbf{Future Directions}
    \begin{itemize}
        \item Extension to domain Y
        \item Scalability improvements
        \item Real-world deployment
    \end{itemize}
\end{frame}

\begin{frame}[standout]
    Questions?

    \vspace{1cm}

    \small
    {{EMAIL}}

    Code: \url{https://github.com/{{GITHUB_REPO}}}
\end{frame}

%% ===================
%% APPENDIX
%% ===================
\appendix

\begin{frame}[allowframebreaks]{References}
    \printbibliography[heading=none]
\end{frame}

\begin{frame}{Supplementary: Additional Results}
    \begin{itemize}
        \item Additional experimental details
        \item Extended ablation studies
        \item Hyperparameter sensitivity analysis
    \end{itemize}
\end{frame}

\begin{frame}{Supplementary: Implementation Details}
    \textbf{Hyperparameters}
    \begin{itemize}
        \item Learning rate: 0.001
        \item Batch size: 64
        \item Training epochs: 100
        \item Optimizer: Adam
    \end{itemize}

    \textbf{Computational Resources}
    \begin{itemize}
        \item GPU: NVIDIA A100
        \item Training time: 4 hours
        \item Memory: 16GB
    \end{itemize}
\end{frame}

\end{document}
