% ICLR Conference Paper Template
% International Conference on Learning Representations
% OpenReview submission format

\documentclass{article}

% Required ICLR style file
\usepackage{iclr2025_conference}      % Update year as needed
% Options:
% \usepackage[final]{iclr2025_conference}     % Final (camera-ready) version
% \usepackage[preprint]{iclr2025_conference}  % Preprint (arXiv) version

% Required packages
\usepackage[utf8]{inputenc}
\usepackage[T1]{fontenc}
\usepackage{hyperref}
\usepackage{url}
\usepackage{booktabs}                 % Better tables
\usepackage{amsfonts}
\usepackage{amsmath}
\usepackage{amssymb}
\usepackage{nicefrac}                 % Compact fractions
\usepackage{microtype}                % Better typography
\usepackage{xcolor}
\usepackage{graphicx}
\usepackage{subcaption}               % Subfigures
\usepackage{algorithm}
\usepackage{algorithmic}

% Custom commands (common in ML papers)
\newcommand{\R}{\mathbb{R}}
\newcommand{\E}{\mathbb{E}}
\newcommand{\Prob}{\mathbb{P}}
\newcommand{\argmax}{\operatorname{argmax}}
\newcommand{\argmin}{\operatorname{argmin}}

%==============================================================================
% DOCUMENT METADATA
%==============================================================================

\title{{{TITLE}}}

% Authors for initial submission (anonymous)
\author{Anonymous}

% Authors for camera-ready (uncomment and fill in):
% \author{
%     {{AUTHOR_1}} \\
%     {{AFFILIATION_1}} \\
%     \texttt{{{EMAIL_1}}} \\
%     \And
%     {{AUTHOR_2}} \\
%     {{AFFILIATION_2}} \\
%     \texttt{{{EMAIL_2}}} \\
%     \And
%     {{AUTHOR_3}} \\
%     {{AFFILIATION_3}} \\
%     \texttt{{{EMAIL_3}}}
% }

% For more authors, use \AND between author blocks:
% \AND is used for a new row of authors

\iclrfinalcopy  % Uncomment for camera-ready version

\begin{document}

\maketitle

%==============================================================================
% ABSTRACT
%==============================================================================

\begin{abstract}
{{ABSTRACT}}
\end{abstract}

%==============================================================================
% INTRODUCTION
%==============================================================================

\section{Introduction}
\label{sec:introduction}

{{INTRODUCTION}}

% ICLR papers typically include:
% - Clear problem statement
% - Summary of contributions (often as a bulleted list)
% - Brief overview of approach
% - Paper organization (optional)

\textbf{Contributions.} Our main contributions are:
\begin{itemize}
    \item {{CONTRIBUTION_1}}
    \item {{CONTRIBUTION_2}}
    \item {{CONTRIBUTION_3}}
\end{itemize}

%==============================================================================
% RELATED WORK
%==============================================================================

\section{Related Work}
\label{sec:related}

{{RELATED_WORK}}

\paragraph{{{RELATED_TOPIC_1}}}
{{RELATED_TOPIC_1_CONTENT}}

\paragraph{{{RELATED_TOPIC_2}}}
{{RELATED_TOPIC_2_CONTENT}}

%==============================================================================
% BACKGROUND / PRELIMINARIES
%==============================================================================

\section{Background}
\label{sec:background}

{{BACKGROUND}}

\subsection{Problem Formulation}
\label{sec:problem}

{{PROBLEM_FORMULATION}}

\subsection{Notation}
\label{sec:notation}

{{NOTATION}}

%==============================================================================
% METHOD / APPROACH
%==============================================================================

\section{Method}
\label{sec:method}

{{METHOD_OVERVIEW}}

\subsection{{{METHOD_SECTION_1}}}
\label{sec:method1}

{{METHOD_SECTION_1_CONTENT}}

\subsection{{{METHOD_SECTION_2}}}
\label{sec:method2}

{{METHOD_SECTION_2_CONTENT}}

\subsection{{{METHOD_SECTION_3}}}
\label{sec:method3}

{{METHOD_SECTION_3_CONTENT}}

% Example algorithm:
% \begin{algorithm}
%     \caption{{{ALGORITHM_NAME}}}
%     \label{alg:method}
%     \begin{algorithmic}[1]
%         \REQUIRE Input data $\mathcal{D}$, hyperparameters $\theta$
%         \ENSURE Trained model $f$
%         \STATE Initialize model parameters
%         \FOR{epoch $= 1$ to $T$}
%             \FOR{batch $\mathcal{B}$ in $\mathcal{D}$}
%                 \STATE Compute loss $\mathcal{L}(\mathcal{B}; \theta)$
%                 \STATE Update parameters via gradient descent
%             \ENDFOR
%         \ENDFOR
%         \RETURN $f$
%     \end{algorithmic}
% \end{algorithm}

%==============================================================================
% THEORETICAL ANALYSIS (optional)
%==============================================================================

\section{Theoretical Analysis}
\label{sec:theory}

{{THEORETICAL_ANALYSIS}}

%==============================================================================
% EXPERIMENTS
%==============================================================================

\section{Experiments}
\label{sec:experiments}

{{EXPERIMENTS_OVERVIEW}}

\subsection{Experimental Setup}
\label{sec:setup}

\paragraph{Datasets}
{{DATASETS}}

\paragraph{Baselines}
{{BASELINES}}

\paragraph{Implementation Details}
{{IMPLEMENTATION_DETAILS}}

\subsection{Main Results}
\label{sec:results}

{{MAIN_RESULTS}}

% Example results table:
% \begin{table}[t]
%     \caption{{{TABLE_1_CAPTION}}}
%     \label{tab:main_results}
%     \centering
%     \begin{tabular}{lccc}
%         \toprule
%         Method & Dataset 1 & Dataset 2 & Dataset 3 \\
%         \midrule
%         Baseline 1 & 85.2 & 78.4 & 91.1 \\
%         Baseline 2 & 87.1 & 80.2 & 92.3 \\
%         \textbf{Ours} & \textbf{89.5} & \textbf{83.7} & \textbf{94.2} \\
%         \bottomrule
%     \end{tabular}
% \end{table}

\subsection{Ablation Studies}
\label{sec:ablation}

{{ABLATION_STUDIES}}

\subsection{Analysis}
\label{sec:analysis}

{{ANALYSIS}}

%==============================================================================
% DISCUSSION
%==============================================================================

\section{Discussion}
\label{sec:discussion}

{{DISCUSSION}}

\paragraph{Limitations}
{{LIMITATIONS}}

\paragraph{Broader Impact}
{{BROADER_IMPACT}}

%==============================================================================
% CONCLUSION
%==============================================================================

\section{Conclusion}
\label{sec:conclusion}

{{CONCLUSION}}

%==============================================================================
% ACKNOWLEDGMENTS (only in camera-ready)
%==============================================================================

% Uncomment for camera-ready version:
% \section*{Acknowledgments}
% {{ACKNOWLEDGMENTS}}

%==============================================================================
% REFERENCES
%==============================================================================

\bibliography{references}
\bibliographystyle{iclr2025_conference}

%==============================================================================
% APPENDIX
%==============================================================================

\appendix

\section{Additional Experimental Details}
\label{app:details}

{{APPENDIX_DETAILS}}

\section{Additional Results}
\label{app:results}

{{APPENDIX_RESULTS}}

\section{Proofs}
\label{app:proofs}

{{APPENDIX_PROOFS}}

\end{document}
