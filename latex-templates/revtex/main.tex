% RevTeX 4.2 Template for APS/AIP Physics Journals
% Physical Review Letters, Physical Review A-E, Physical Review X, etc.
% RevTeX is included in most TeX distributions

\documentclass[
    aps,                              % American Physical Society
    prl,                              % Physical Review Letters (change to pra, prb, prc, prd, pre, prx as needed)
    twocolumn,                        % Two-column format
    superscriptaddress,               % Affiliations as superscripts
    showpacs,                         % Show PACS numbers
    showkeys,                         % Show keywords
    reprint                           % Reprint format
]{revtex4-2}

% Alternative journal options:
% aps,prl  - Physical Review Letters
% aps,pra  - Physical Review A (Atomic, Molecular, Optical)
% aps,prb  - Physical Review B (Condensed Matter)
% aps,prc  - Physical Review C (Nuclear)
% aps,prd  - Physical Review D (Particles, Fields, Gravitation)
% aps,pre  - Physical Review E (Statistical, Nonlinear, Soft Matter)
% aps,prx  - Physical Review X
% aip,jcp  - Journal of Chemical Physics
% aip,apl  - Applied Physics Letters

% Required packages
\usepackage[utf8]{inputenc}
\usepackage{amsmath,amssymb,amsfonts}
\usepackage{graphicx}
\usepackage{hyperref}
\usepackage{bm}                       % Bold math
\usepackage{siunitx}                  % SI units
\usepackage{physics}                  % Physics notation (optional)

% Hyperref setup
\hypersetup{
    colorlinks=true,
    linkcolor=blue,
    citecolor=blue,
    urlcolor=blue
}

%==============================================================================
% DOCUMENT METADATA
%==============================================================================

\begin{document}

\preprint{APS/123-QED}                % Preprint number (optional)

\title{{{TITLE}}}

% Authors with affiliations
\author{{{AUTHOR_1}}}
\email{{{AUTHOR_1_EMAIL}}}
\affiliation{{{AFFILIATION_1}}}

\author{{{AUTHOR_2}}}
\affiliation{{{AFFILIATION_2}}}

\author{{{AUTHOR_3}}}
\affiliation{{{AFFILIATION_1}}}
\affiliation{{{AFFILIATION_2}}}       % Multiple affiliations

% Collaboration (for large collaborations)
% \collaboration{The Example Collaboration}

\date{\today}

%==============================================================================
% ABSTRACT
%==============================================================================

\begin{abstract}
{{ABSTRACT}}
\end{abstract}

%==============================================================================
% PACS AND KEYWORDS
%==============================================================================

% PACS numbers (Physics and Astronomy Classification Scheme)
% Find codes at: https://journals.aps.org/PACS
\pacs{{{PACS_NUMBERS}}}
% Example: 03.67.-a, 42.50.Dv, 73.21.La

\keywords{{{KEYWORDS}}}

\maketitle

%==============================================================================
% INTRODUCTION
%==============================================================================

\section{Introduction}
\label{sec:introduction}

{{INTRODUCTION}}

%==============================================================================
% THEORETICAL FRAMEWORK / MODEL
%==============================================================================

\section{Theoretical Framework}
\label{sec:theory}

{{THEORY}}

% Example equation:
% The Hamiltonian of the system is given by
% \begin{equation}
%     \hat{H} = \sum_{i} \frac{\hat{p}_i^2}{2m} + \sum_{i<j} V(|\mathbf{r}_i - \mathbf{r}_j|),
%     \label{eq:hamiltonian}
% \end{equation}
% where $\hat{p}_i$ is the momentum operator and $V$ is the interaction potential.

\subsection{{{THEORY_SUBSECTION_1}}}

{{THEORY_SUBSECTION_1_CONTENT}}

\subsection{{{THEORY_SUBSECTION_2}}}

{{THEORY_SUBSECTION_2_CONTENT}}

%==============================================================================
% EXPERIMENTAL METHODS / NUMERICAL METHODS
%==============================================================================

\section{Methods}
\label{sec:methods}

{{METHODS}}

\subsection{Experimental Setup}

{{EXPERIMENTAL_SETUP}}

\subsection{Measurement Procedure}

{{MEASUREMENT_PROCEDURE}}

%==============================================================================
% RESULTS
%==============================================================================

\section{Results}
\label{sec:results}

{{RESULTS}}

% Example figure reference:
% As shown in Fig.~\ref{fig:results}, the measured values agree with theoretical predictions.

% Example table reference:
% The numerical results are summarized in Table~\ref{tab:data}.

\subsection{{{RESULTS_SUBSECTION_1}}}

{{RESULTS_SUBSECTION_1_CONTENT}}

\subsection{{{RESULTS_SUBSECTION_2}}}

{{RESULTS_SUBSECTION_2_CONTENT}}

%==============================================================================
% DISCUSSION
%==============================================================================

\section{Discussion}
\label{sec:discussion}

{{DISCUSSION}}

%==============================================================================
% CONCLUSION
%==============================================================================

\section{Conclusion}
\label{sec:conclusion}

{{CONCLUSION}}

%==============================================================================
% ACKNOWLEDGMENTS
%==============================================================================

\begin{acknowledgments}
{{ACKNOWLEDGMENTS}}
% Example: This work was supported by the National Science Foundation under Grant No. PHY-XXXXXXX.
\end{acknowledgments}

%==============================================================================
% APPENDICES (optional)
%==============================================================================

\appendix

\section{{{APPENDIX_A_TITLE}}}
\label{app:a}

{{APPENDIX_A_CONTENT}}

%==============================================================================
% REFERENCES
%==============================================================================

\bibliography{references}

%==============================================================================
% FIGURES
%==============================================================================

% Figures in RevTeX are typically placed at the end for submission
% They will be positioned appropriately in the final publication

\begin{figure}
    \centering
    % \includegraphics[width=\columnwidth]{figure1.pdf}
    \caption{{{FIGURE_1_CAPTION}}}
    \label{fig:figure1}
\end{figure}

\begin{figure}
    \centering
    % \includegraphics[width=\columnwidth]{figure2.pdf}
    \caption{{{FIGURE_2_CAPTION}}}
    \label{fig:figure2}
\end{figure}

%==============================================================================
% TABLES
%==============================================================================

\begin{table}
    \caption{{{TABLE_1_CAPTION}}}
    \label{tab:table1}
    \begin{ruledtabular}
    \begin{tabular}{lccc}
        Parameter & Value & Uncertainty & Unit \\
        \hline
        {{TABLE_1_CONTENT}}
    \end{tabular}
    \end{ruledtabular}
\end{table}

\end{document}
