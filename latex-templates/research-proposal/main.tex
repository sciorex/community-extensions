%% Research Proposal Template
%% For grant applications, PhD proposals, or project proposals
%%
%% Compile with: pdflatex main.tex && bibtex main && pdflatex main.tex && pdflatex main.tex

\documentclass[11pt,letterpaper]{article}

%% Packages
\usepackage[utf8]{inputenc}
\usepackage[T1]{fontenc}
\usepackage{lmodern}
\usepackage[margin=1in]{geometry}
\usepackage{setspace}
\usepackage{graphicx}
\usepackage{amsmath}
\usepackage{amssymb}
\usepackage{booktabs}
\usepackage{enumitem}
\usepackage{hyperref}
\usepackage{xcolor}
\usepackage{fancyhdr}
\usepackage{titlesec}
\usepackage{natbib}

%% Line spacing
\onehalfspacing

%% Hyperref setup
\hypersetup{
    colorlinks=true,
    linkcolor=blue,
    citecolor=blue,
    urlcolor=blue
}

%% Header/Footer
\pagestyle{fancy}
\fancyhf{}
\fancyhead[L]{\small {{TITLE}}}
\fancyhead[R]{\small {{PI_NAME}}}
\fancyfoot[C]{\thepage}

%% Section formatting
\titleformat{\section}{\large\bfseries}{\thesection.}{1em}{}
\titleformat{\subsection}{\normalsize\bfseries}{\thesubsection}{1em}{}

\begin{document}

%% ===================
%% TITLE PAGE
%% ===================
\begin{titlepage}
\centering
\vspace*{1in}

{\LARGE\bfseries {{TITLE}}}

\vspace{0.5in}

{\large Research Proposal}

\vspace{1in}

{\Large {{PI_NAME}}}

\vspace{0.25in}

{{PI_AFFILIATION}} \\
{{PI_DEPARTMENT}} \\
{{PI_EMAIL}}

\vspace{1in}

\begin{tabular}{ll}
\textbf{Funding Agency:} & {{FUNDING_AGENCY}} \\
\textbf{Project Duration:} & {{DURATION}} \\
\textbf{Requested Budget:} & {{BUDGET}} \\
\textbf{Submission Date:} & {{DATE}}
\end{tabular}

\vfill

\end{titlepage}

%% ===================
%% EXECUTIVE SUMMARY
%% ===================
\section{Executive Summary}
\label{sec:summary}

{{ABSTRACT}}

\vspace{1em}
\noindent\textbf{Keywords:} {{KEYWORDS}}

%% ===================
%% INTRODUCTION
%% ===================
\section{Introduction and Background}
\label{sec:introduction}

\subsection{Problem Statement}

Clearly articulate the research problem or challenge that this project addresses. Explain why this problem is significant and timely \citep{chen2023foundations}.

\subsection{Motivation}

Describe the motivation behind this research. What gaps in current knowledge or practice does this project aim to address? Why is solving this problem important for the field and society?

\subsection{Significance}

Explain the potential impact of this research. How will the outcomes advance the field? What practical applications might emerge?

%% ===================
%% RESEARCH OBJECTIVES
%% ===================
\section{Research Objectives}
\label{sec:objectives}

\subsection{Primary Objective}

State the main goal of the proposed research clearly and concisely.

\subsection{Specific Aims}

\begin{enumerate}[label=\textbf{Aim \arabic*:}]
    \item {{AIM_1}} -- Describe the first specific aim and its expected outcomes.
    \item {{AIM_2}} -- Describe the second specific aim and its expected outcomes.
    \item {{AIM_3}} -- Describe the third specific aim and its expected outcomes.
\end{enumerate}

\subsection{Research Questions}

\begin{itemize}
    \item RQ1: What is the relationship between X and Y?
    \item RQ2: How does intervention Z affect outcome W?
    \item RQ3: What are the key factors influencing phenomenon P?
\end{itemize}

%% ===================
%% LITERATURE REVIEW
%% ===================
\section{Literature Review}
\label{sec:literature}

\subsection{Theoretical Framework}

Present the theoretical foundations upon which this research is built. Discuss relevant theories, models, or frameworks that inform the proposed approach \citep{smith2022theoretical}.

\subsection{Current State of Knowledge}

Review the existing literature on the topic. Identify key findings, methodological approaches, and remaining questions \citep{johnson2024review}.

\subsection{Research Gap}

Clearly identify the gap in the current literature that this proposal addresses. Explain how the proposed research will contribute new knowledge to the field.

%% ===================
%% METHODOLOGY
%% ===================
\section{Methodology}
\label{sec:methodology}

\subsection{Research Design}

Describe the overall research design and justify its appropriateness for addressing the research questions. Explain whether the approach is quantitative, qualitative, or mixed-methods \citep{williams2023methods}.

\subsection{Data Collection}

\subsubsection{Participants/Samples}

Describe the target population, sampling strategy, and sample size justification.

\subsubsection{Instruments and Materials}

Detail the instruments, tools, or materials that will be used for data collection.

\subsubsection{Procedures}

Provide a step-by-step description of the data collection procedures.

\subsection{Data Analysis}

\subsubsection{Analysis Plan}

Describe the analytical methods that will be used to address each research question.

\subsubsection{Statistical Approaches}

If applicable, specify the statistical tests or computational methods to be employed.

\subsection{Ethical Considerations}

Address any ethical issues related to the research, including human subjects protections, data privacy, and institutional review board approval.

%% ===================
%% TIMELINE
%% ===================
\section{Project Timeline}
\label{sec:timeline}

\begin{table}[h]
\centering
\caption{Project milestones and timeline.}
\label{tab:timeline}
\begin{tabular}{@{}llc@{}}
\toprule
\textbf{Phase} & \textbf{Activities} & \textbf{Duration} \\
\midrule
Phase 1 & Literature review, protocol development & Months 1--3 \\
Phase 2 & Data collection, preliminary analysis & Months 4--9 \\
Phase 3 & Full data analysis, interpretation & Months 10--15 \\
Phase 4 & Writing, dissemination, final report & Months 16--18 \\
\bottomrule
\end{tabular}
\end{table}

\subsection{Key Milestones}

\begin{itemize}
    \item Month 3: Complete literature review and finalize methodology
    \item Month 6: Complete IRB approval and begin data collection
    \item Month 12: Complete data collection
    \item Month 15: Complete data analysis
    \item Month 18: Submit final report and publications
\end{itemize}

%% ===================
%% EXPECTED OUTCOMES
%% ===================
\section{Expected Outcomes}
\label{sec:outcomes}

\subsection{Deliverables}

\begin{itemize}
    \item Peer-reviewed publications (target: 2--3 articles)
    \item Conference presentations (target: 3--4 presentations)
    \item Final project report
    \item Open-source tools/datasets (if applicable)
\end{itemize}

\subsection{Anticipated Impact}

Describe how the expected outcomes will advance the field and potentially influence practice or policy.

%% ===================
%% BUDGET JUSTIFICATION
%% ===================
\section{Budget Justification}
\label{sec:budget}

\begin{table}[h]
\centering
\caption{Budget summary.}
\label{tab:budget}
\begin{tabular}{@{}lr@{}}
\toprule
\textbf{Category} & \textbf{Amount} \\
\midrule
Personnel (PI, Research Assistants) & \$XX,XXX \\
Equipment and Supplies & \$XX,XXX \\
Travel (Conferences, Field Work) & \$XX,XXX \\
Participant Compensation & \$XX,XXX \\
Indirect Costs & \$XX,XXX \\
\midrule
\textbf{Total} & \${{BUDGET}} \\
\bottomrule
\end{tabular}
\end{table}

\subsection{Personnel}

Justify the personnel costs, including the time commitment of the PI and any research staff.

\subsection{Equipment and Supplies}

Describe essential equipment and supplies needed for the project.

\subsection{Travel}

Justify travel expenses for conferences, collaboration, or field work.

%% ===================
%% BROADER IMPACTS
%% ===================
\section{Broader Impacts}
\label{sec:impacts}

\subsection{Educational Benefits}

Describe opportunities for student training, mentorship, and educational outreach.

\subsection{Societal Benefits}

Explain how this research may benefit society beyond the immediate scientific community.

\subsection{Diversity and Inclusion}

Address how the project will promote diversity and inclusion in the research team and beneficiaries.

%% ===================
%% QUALIFICATIONS
%% ===================
\section{Qualifications of the Research Team}
\label{sec:qualifications}

\subsection{Principal Investigator}

Briefly describe the PI's relevant expertise, prior work, and qualifications to lead this project.

\subsection{Collaborators}

List any collaborators and their roles in the project.

%% ===================
%% REFERENCES
%% ===================
\bibliographystyle{apalike}
\bibliography{references}

\end{document}
