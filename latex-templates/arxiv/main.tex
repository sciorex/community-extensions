\documentclass{article}

% arXiv-compatible packages
\usepackage{arxiv}
\usepackage[utf8]{inputenc}
\usepackage[T1]{fontenc}
\usepackage{hyperref}
\usepackage{url}
\usepackage{booktabs}
\usepackage{amsfonts}
\usepackage{amsmath}
\usepackage{amssymb}
\usepackage{nicefrac}
\usepackage{microtype}
\usepackage{lipsum}
\usepackage{graphicx}

\title{{{TITLE}}}

\author{
  {{AUTHOR}} \\
  Your Institution\\
  Department Name\\
  \texttt{{{EMAIL}}} \\
}

\begin{document}

\maketitle

\begin{abstract}
This document is a template for arXiv preprint papers. Replace this text with your abstract. The abstract should provide a concise summary of your work, including the problem, approach, and main findings.
\end{abstract}

\keywords{keyword1 \and keyword2 \and keyword3}

\section{Introduction}
This is the introduction section. Replace this text with your introduction content.

\section{Related Work}
Discuss related work here. You can cite references like this~\cite{example2024}.

\section{Methodology}
Describe your methodology in this section.

\subsection{Subsection Example}
You can use subsections to organize your content.

\subsubsection{Subsubsection Example}
Even deeper organization if needed.

\section{Experiments}
Describe your experimental setup and procedures.

\section{Results}
Present your results here. You can include figures and tables.

\section{Discussion}
Discuss the implications of your results.

\section{Conclusion}
Summarize your findings and contributions.

\section*{Acknowledgments}
Optional acknowledgments section.

\bibliographystyle{plain}
\bibliography{references}

\end{document}
