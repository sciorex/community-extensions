% American Chemical Society (ACS) Article Template
% For submissions to ACS journals
%
% Compile with: pdflatex main.tex
%               bibtex main
%               pdflatex main.tex
%               pdflatex main.tex

\documentclass[journal=jacsat,manuscript=article]{achemso}

% ACS required packages
\usepackage[version=4]{mhchem}  % Chemical formulas
\usepackage{chemfig}            % Chemical structures
\usepackage{siunitx}            % SI units
\usepackage{graphicx}
\usepackage{booktabs}
\usepackage{multirow}
\usepackage{amsmath}
\usepackage{amssymb}

% Configure siunitx
\sisetup{
    separate-uncertainty = true,
    multi-part-units = single,
}

% Author information
\author{{{AUTHOR_1}}}
\email{{{AUTHOR_1_EMAIL}}}
\affiliation{{{AUTHOR_1_AFFILIATION}}, {{AUTHOR_1_ADDRESS}}}

\author{{{AUTHOR_2}}}
\affiliation{{{AUTHOR_2_AFFILIATION}}, {{AUTHOR_2_ADDRESS}}}

\author{{{CORRESPONDING_AUTHOR}}}
\email{{{CORRESPONDING_EMAIL}}}
\phone{{{CORRESPONDING_PHONE}}}
\affiliation{{{CORRESPONDING_AFFILIATION}}}
\alsoaffiliation{{{CORRESPONDING_ALSO_AFFILIATION}}}

\title{{{TITLE}}}

\begin{document}

\begin{abstract}
{{ABSTRACT}}
\end{abstract}

\section{Introduction}

The development of novel materials and chemical processes is essential for addressing global challenges in energy, health, and sustainability.\cite{smith2022sustainable} In this work, we present a new approach to {{RESEARCH_TOPIC}} that offers significant improvements over existing methods.

The synthesis of functional materials requires precise control over reaction conditions and molecular architecture.\cite{johnson2021synthesis} Previous studies have demonstrated the importance of {{KEY_FACTOR}} in determining the final properties of these systems.\cite{williams2020characterization}

Our approach addresses several key challenges:
\begin{itemize}
    \item Improved selectivity in the synthesis process
    \item Enhanced stability under operating conditions
    \item Scalable production methods
    \item Reduced environmental impact
\end{itemize}

\section{Experimental Section}

\subsection{Materials}

All chemicals were purchased from commercial suppliers and used without further purification unless otherwise noted. {{CHEMICAL_1}} (\SI{99}{\percent}, Sigma-Aldrich), {{CHEMICAL_2}} (\SI{98}{\percent}, Alfa Aesar), and {{SOLVENT}} (HPLC grade, Fisher Scientific) were used as received.

\subsection{Synthesis of {{COMPOUND_NAME}}}

{{COMPOUND_NAME}} was synthesized according to the following procedure. In a typical experiment, {{CHEMICAL_1}} (\SI{1.0}{\gram}, \SI{10}{\milli\mol}) was dissolved in {{SOLVENT}} (\SI{50}{\milli\liter}) in a round-bottom flask equipped with a magnetic stir bar. The solution was heated to \SI{80}{\celsius} under nitrogen atmosphere.

{{CHEMICAL_2}} (\SI{1.2}{\gram}, \SI{12}{\milli\mol}) was added dropwise over \SI{30}{\minute}. The reaction mixture was stirred for \SI{24}{\hour} at \SI{80}{\celsius}. Upon completion, the mixture was cooled to room temperature and the product was isolated by filtration.

\textbf{Yield:} \SI{85}{\percent}. \textbf{Melting point:} \SI{145}{\celsius}.

\textsuperscript{1}\textbf{H NMR} (\SI{400}{\mega\hertz}, CDCl\textsubscript{3}): $\delta$ 7.85 (d, $J$ = \SI{8.0}{\hertz}, 2H), 7.42 (t, $J$ = \SI{7.5}{\hertz}, 2H), 7.35 (t, $J$ = \SI{7.2}{\hertz}, 1H), 3.92 (s, 3H).

\textbf{HRMS} (ESI): $m/z$ calcd for \ce{C_{x}H_{y}N_{z}O_{w}} [M+H]\textsuperscript{+} {{CALC_MASS}}; found {{FOUND_MASS}}.

\subsection{Characterization}

\subsubsection{Nuclear Magnetic Resonance (NMR) Spectroscopy}

\textsuperscript{1}H and \textsuperscript{13}C NMR spectra were recorded on a Bruker Avance III \SI{400}{\mega\hertz} spectrometer at \SI{298}{\kelvin}. Chemical shifts are reported in ppm relative to tetramethylsilane (TMS).

\subsubsection{X-ray Diffraction (XRD)}

Powder X-ray diffraction patterns were collected on a Rigaku MiniFlex diffractometer using Cu K$\alpha$ radiation ($\lambda$ = \SI{1.5406}{\angstrom}). Data were collected in the 2$\theta$ range of \SI{5}{\degree}--\SI{60}{\degree} with a step size of \SI{0.02}{\degree}.

\subsubsection{Scanning Electron Microscopy (SEM)}

SEM images were obtained using a JEOL JSM-7500F field emission scanning electron microscope operated at \SI{5}{\kilo\volt}.

\subsubsection{Thermogravimetric Analysis (TGA)}

TGA measurements were performed on a TA Instruments Q500 under nitrogen flow (\SI{50}{\milli\liter\per\minute}) with a heating rate of \SI{10}{\celsius\per\minute} from \SI{25}{\celsius} to \SI{800}{\celsius}.

\section{Results and Discussion}

\subsection{Structural Characterization}

The synthesized {{COMPOUND_NAME}} was characterized by multiple analytical techniques. Figure~\ref{fig:xrd} shows the XRD pattern, confirming the crystalline nature of the material.

\begin{figure}[htbp]
    \centering
    % \includegraphics[width=0.8\linewidth]{figures/xrd_pattern.pdf}
    \caption{Powder XRD pattern of {{COMPOUND_NAME}}. The sharp peaks indicate high crystallinity.}
    \label{fig:xrd}
\end{figure}

The NMR analysis confirms the expected structure with all peaks assigned to the corresponding protons (see Supporting Information).

\subsection{Performance Evaluation}

The catalytic activity of {{COMPOUND_NAME}} was evaluated under standard conditions. Table~\ref{tab:performance} summarizes the results.

\begin{table}[htbp]
    \centering
    \caption{Catalytic performance of {{COMPOUND_NAME}} compared to reference materials.}
    \label{tab:performance}
    \begin{tabular}{lccc}
        \toprule
        \textbf{Catalyst} & \textbf{Conversion (\%)} & \textbf{Selectivity (\%)} & \textbf{TOF (h\textsuperscript{-1})} \\
        \midrule
        Reference A & 65 & 78 & 120 \\
        Reference B & 72 & 82 & 145 \\
        {{COMPOUND_NAME}} & \textbf{89} & \textbf{94} & \textbf{210} \\
        \bottomrule
    \end{tabular}
\end{table}

The superior performance of our material can be attributed to the unique structural features that provide:
\begin{enumerate}
    \item Enhanced active site accessibility
    \item Optimal electronic properties
    \item Improved thermal stability
\end{enumerate}

\subsection{Mechanistic Studies}

Kinetic studies were performed to elucidate the reaction mechanism. The reaction follows first-order kinetics with respect to substrate concentration:

\begin{equation}
    r = k[S]
\end{equation}

where $r$ is the reaction rate, $k$ is the rate constant, and $[S]$ is the substrate concentration.

The activation energy was determined from Arrhenius analysis:

\begin{equation}
    E_a = \SI{45.2 \pm 2.1}{\kilo\joule\per\mol}
\end{equation}

This value is consistent with the proposed mechanism involving rate-limiting substrate adsorption.

\subsection{Stability Studies}

Long-term stability tests showed that {{COMPOUND_NAME}} maintains \SI{95}{\percent} of its initial activity after 100 hours of continuous operation. This excellent stability is attributed to the robust structural framework.

\section{Conclusions}

We have developed a novel synthesis method for {{COMPOUND_NAME}} that demonstrates exceptional performance in {{APPLICATION}}. The key findings include:

\begin{itemize}
    \item High yield synthesis (\SI{85}{\percent}) under mild conditions
    \item Superior catalytic activity compared to existing materials
    \item Excellent long-term stability
    \item Scalable production potential
\end{itemize}

This work provides a foundation for further development of advanced materials for sustainable chemical processes.

\begin{acknowledgement}
{{ACKNOWLEDGMENTS}}
\end{acknowledgement}

\begin{suppinfo}
Supporting Information Available: Additional characterization data including NMR spectra, mass spectra, and additional figures. This material is available free of charge via the Internet at http://pubs.acs.org.
\end{suppinfo}

\bibliography{references}

\end{document}
